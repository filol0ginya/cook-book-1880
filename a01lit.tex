%литература после первой главы
\subsection*{Литература}\addcontentsline{toc}{subsubsection}{Литература}
Историки, полагая, что Средние века занимались теологией, самое большее психологией и космологией, долго не предполагали искать в их наследии эстетику, так же долго не было литературы о средневековой эстетике. О ней не хватало информации в общих исследованиях истории философии. А специальные исследования истории эстетики перескакивали период Средневековья. Изложив античную эстетику, переходили непосредственно к изложению эстетики Нового времени.

Действительно, писатели Средневековья не оставили трактатов по эстетике, но в теологических, психологических, космологических трактатах они создавали эстетические основы или делали эстетические выводы, выражали определённое понятие красоты и искусства. Много текстов, интересующих историка эстетики содержат издательства J.~P.~Migne \textit{Patrologia Graeca} (цит. ниже как \textit{P.~G.}) в 161 томах и \textit{Patrologia Latina} (цит. \textit{P.~L.}) в 221 томах, а также позднейшие, преимущественно лучшие издания средневековых писателей. Часть их произведений является ещё не изданными рукописями.

Первыми работами по средневековой эстетике были монографические исследования конца XIX~в. об Августине, потом о Фоме. Их было мало и они охватывали незначительную часть темы. В то время как после Второй мировой войны сразу появилось исследование всего наследия Средневековья: Edgar  D~e~~B~r~u~y~n~e,  \textit{Études d’Esthétique Médiévale}, в 3-х томах, изд.~Гентского университета, 1946. Благодаря работе одного человека средневековые материалы по истории эстетики стали собраны полней, чем античные материалы. Они более специально разработаны, но не объединены ещё из сотен монографических работ. На материале, который собрал де Брюйне, основана в немалой степени данная работа: эти материалы требовали скорей сокращений и отбора, потому что вместе с важными для эстетики текстами включают в себя множество текстов несущественных. Кроме издания этих Работ, послуживших источником, де Брюйне дважды систематически изложил средневековую эстетику: по-французски: \textit{Esthétique du Moyen Age}, Лёвен 1947 и по-фламандски: \textit{Geschiedenis van de Aesthetica de Middeleeuwen}, Антверпен 1951--1955.

Материалы, собранные де Брюйне не охватывают эстетики восточного христианства, а эстетику Запада начинают после Августина. О Боэции де Брюйне пишет в т.~I, стр.~3, о Кассиодоре I, 35, об Исидоре I, 74, об эстетике эпохи Каролингов I, 165, о средневековой поэтике I, 216 и II, 3, о средневековой теории музыки I, 306 и II, 108, о теории пластического искусства I, 243, II, 69 и II, 371, об эстетике мистиков III, 30, об эстетике викторианцев II, 203, о Гильоме~XII, Гильоме Оссерском и о Сумме Александра Гэльского III, 72, о Роберте Гроссетесте III, 121, о Бонавентуре III, 189, об Альберте Великом III, 153, о Фоме Аквинском III, 278, о Витело III, 239.

Де Брюйне пишет в предисловии, что он намеревался дать \emph{un recueil de textes devant servir à l’histoire de l’esthétique médiévale}, но отказался от этого намерения. Тексты он приводит частью в исследовании, частью в сносках, преимущественно в оригинале, иногда с переводом, иногда только по-французски. Кроме этих  работ собраний источников по средневековой эстетике нет. Таким образом, данная работа берёт на себя задачу составления сборника текстов, представляющихся наиболее значимыми, по аналогии с античной эстетикой. Сборник не является полным, но некоторые мысли из области эстетики повторялись средневековыми авторами так часто, что полное собрание перестанет быть полезным из-за своего однообразия; более важным представляется выбор типовых текстов.

Единственный до сих пор большой сборник текстов по средневековой эстетике в итальянском переводе в \textit{Grande Antologia Filosofica}, т.~V, 1954: R.~~M~o~n~t~a~n~o, \textit{L’estetica nel pensiero cristiano}, стр.~207--310.

После работ де Брюйне лучшее синтетическое исследование средневековой эстетики имеет итальянская литература в сборнике \textit{Momenti e problemi di storia dell’estetica}, т.~I, 1959, а именно: Q.~~C~a~t~a~u~d~e~l~l~a,  \textit{Estetica cristiana}, стр.~81--114, а также U.~~E~c~o,  \textit{Sviluppo dell’estetica medievale}, стр.~115--229.

Эта публикация содержит также наиболее полное собрание литературы по предмету исследования (стр. 113--114 и 217--229). Следовало её дополнить главным образом некоторыми работами по истории литературы, музыки и пластических искусств, включающими в себя общие рассмотрение эстетической природы. Вся монографическая литература, касающаяся средневековой эстетики, сильно ограничена, имеет большие пробелы. В данной \emph{Истории} важные позиции приведены в ссылках, в особенности те, которые связаны с эстетикой \emph{Св. Писания} (на стр.~\pageref{sec:pismo_sw}), с эстетикой Отцов Церкви (на стр.~\pageref{sec:greccy}), с византинистикой (на стр.~\pageref{sec:bizantynistyka}), с эстетикой Августина (на стр.~\pageref{subsec:sw_august}, 56, 58, 64, 65), Фомы Аквинского (на стр. 268), с эстетикой пластического искусства (на стр. 157, 159, 160, 162, 163, 167, 170, 184), музыки (на стр. 140), поэзии (на стр. 128, 130).