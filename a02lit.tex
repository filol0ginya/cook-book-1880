%литература после второй главы

\y{Тексты из Св.~Писания}\addcontentsline{toc}{subsubsection}{\refname}


\begin{theorem}\label{cyt:byt1:31}%1
\textnormal{\textgreek{Ka`i e\~>iden <o je`os t`a pant\'a, <'osa >epo\'ihsen ka`i >ido`u kal`a li\'an.}
(Viditque Deus cuncta, quae fecerat, et erant valde bona).
Красота мира:
<<И увидел Бог всё, что Он создал, и вот, хорошо весьма>>.
Бытие, 1:31}
\end{theorem}

\begin{theorem}\label{cyt:tyk39:21}%2
\textnormal {\textgreek{T`a >'erga kur'iou p'anta <'oti kal`a sf'odra.}
(Opera domini universa bona valde).
<<все дела Господа весьма благотворны>>
EKLEZJASTYK, 39:21 ??? % русский перевод взят из Сираха
}
\end{theorem}

\begin{theorem}\label{cyt:ta3:11}%3
\textnormal {\textgreek{s'un t`a p'anta >epo'ihsen kal`a >en kair\~w| a>uto\~u.}
(Cuncta fecit bona in tempore suo).
<<Всё соделал Он прекрасным в своё время>>.
Екклесиаст, 3:11}
\end{theorem}

\begin{theorem}\label{cyt:ps25:8}%4
\textnormal{\textgreek{k'urie, >hg'aphsa e>upr'epeian o>'ikou sou.}
(Domine, dilexi decorem domus tuae).
<<Господи! возлюбил я обитель дома Твоего>>. 
Псалом, 25:8}
\end{theorem}

\begin{theorem}\label{cyt:ps95:5}%5
\textnormal{\textgreek{>exomol'oghsis ka`i <wrai'oths >en'wpion a>uto\~u.}
(Majestas et decor praecedunt eum)
<<Ибо все боги народов — идолы, а Господь небеса сотворил>>.
Псалом, 95:5}
\end{theorem}

\begin{theorem}\label{cyt:prem13:5}%6
\textnormal{\textgreek{>Ek g`ar meg'ejous ka`i kallon\~hs ktism'atwn >anal'ogws <o genesiourg`os a>ut\~wn jewre\~itai.}
(A magnitudine enim speciei et creaturae cognoscibiliter poterit creator horum videri).
Красота мира указывает Творца
<<ибо от величия красоты созданий сравнительно познается Виновник бытия их>>.
Книга Премудрости, 13:5}
\end{theorem}
	
\begin{theorem}\label{cyt:prem11:21}%7
\textnormal{\textgreek{p'anta m'etrw| ka`i >arijm\~w| ka`i stajm\~w| di'etaxas.}
(Omnia in mensura et numero et pondere disposuisti).
ŚWIAT ZAWDZIĘCZA SWE PIĘKNO MIERZE, LICZBIE I WADZE
<<Ты всё расположил мерою, числом и весом>>.
Книга Премудрости, 11:21}
\end{theorem}

\begin{theorem}\label{cyt:tyk1:9}%8
\textnormal{\textgreek{kuri'os a>ut`os >'ektisen a>ut'hn ka`i e\~>ide kai >exer'ijmhsen a>ut'hn.}
(Ille creavit illam in Spiritu sancto et vidit, et dinumeravit, et mensus est).
<<Он произвёл её (мудрость) и видел и измерил её и излил её на все дела Свои>>. 
EKLEZJASTYK, 1:9 % русский перевод взят из Сираха
}
\end{theorem}

\begin{theorem}\label{cyt:prit31:30}%9
\textnormal{\textgreek{yeude\~is >areske'iai, ka`i m'ataion k'allos.}
(Fallax gratia et vana est pulchritudo)
Ничтожность красоты
<<Миловидность обманчива и красота суетна>>. 
Книга притчей Соломона, 31:30}
\end{theorem}

\begin{theorem}\label{cyt:prit2:14}%10
\textnormal{{Ostende mihi faciem tuam et auditum fac mihi vocem tuam, quoniam vox tua s~u~a~v~i~s est mihi et facies tua s~p~e~c~i~o~s~a.}
Красота Бога % тут про женщину, вообще-то
<<покажи мне лице твое, дай мне услышать голос твой, потому что голос твой сладок и лице твое приятно>>.
Песнь песней, 2:14}
\end{theorem}

\begin{theorem}\label{cyt:gaj5}%11
\textnormal{\textgreek{T`o >ag'enhton ka`i je\~ion <or\~an \ldots t`o kre'itton m`en agajo\~u, k'allion d`e kalo\~u.}
<<и это узренье Бога я ставлю выше всех прочих вещей, для каждого (из нас) и вместе для всех людей>>.
Филон Александрийский, <<О посольстве к Гаю>>, 5.}
\end{theorem}

\begin{theorem}\label{cyt:ish20:4}%12
\textnormal{\textgreek{O>u poie'hseis seaut\~w| e>'idwlon o>ud`e pant`os <omo'iwma <'osa >en t\~w| o>uram\~w| >'anw kai <'osa >en t\~h| g\~h| k'atw kai <'osa >en to\~is <'udasin <upok'atw t\~hs g\~hs.}
(Non facies tibi sculptile neque omnem similitudinem quae est in coelo desuper et quae in terra deorsum nec eorum quae sunt in aquis sub terra).
Запрет на изображения
<<Не делай себе кумира и никакого изображения того, что на небе вверху, и что на земле внизу, и что в воде ниже земли>>. 
Исход, 20:4}
\end{theorem}

\begin{theorem}\label{cyt:wtor27:15}%13
\textnormal{\textgreek{>epikat'aratos >'anjropos <'ostis poi'hsei glupt`on ka`i qwneut'on, bd'elugma kur'iw|, >'ergon qeir\~wn teqnit\~wn.}
(Maledictus homo, qui facit sculptibile et conflatibile, abominationem Domini, opus manuum artificum).
<<проклят, кто сделает изваянный или литый кумир, мерзость пред Господом, произведение рук художника>>.
Второзаконие, 27:15}
\end{theorem}

\begin{theorem}\label{cyt:mat6:28}%14
\textnormal{Considerate lilia agri quomodo crescunt: non laborant neque nent. Dico autem vobis quoniam nec Salomon in omni gloria sua coopertus est sicut unum ex istis.
Красота природы
<<Посмотрите на полевые лилии, как они растут: ни трудятся, ни прядут; но говорю вам, что и Соломон во всей славе своей не одевался так, как всякая из них>>.
Евангелие от Матфея, 6:28-29 % в оригинале (5:28-29), но там не про лилии, а про вожделение
}
\end{theorem}

\begin{theorem}\label{cyt:rim1:20}%15
\textnormal{Invisibilia enim ipsius a creatura mundi, per ea quae facta sunt, intellecta conspiciuntur.
Бог проявляется в Творении.
<<Ибо невидимое Его, вечная сила Его и Божество, от создания мира через рассматривание творений видимы, так что они безответны>>.
Послание апостола Павла к римлянам, 1:20}
\end{theorem}