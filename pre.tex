%%% русский язык
\usepackage[greek,english,russian]{babel}
\usepackage[or]{teubner} %для греческого (см.выше)
\usepackage{cmap}					% поиск в PDF
\usepackage[T2A]{fontenc}			% кодировка
\usepackage[utf8]{inputenc}			% кодировка исходного текста
\usepackage{indentfirst}            % отступ первой красной строки раздела
\frenchspacing

%\usepackage{lastpage} %показывает последнюю страницу

\usepackage{fancyhdr} % Колонтитулы
 	\pagestyle{fancy}
 	\renewcommand{\headrulewidth}{0.5pt}  % Толщина линейки, отчеркивающей верхний колонтитул
 	%\lfoot{Владислав Татаркевич}
 	%\rfoot{История эстетики}
 	%\rhead{Глава}
 	% \chead{Верхний в центре}
 	%\lhead{\copyright Жарова~А.~В., перевод}
 	% \cfoot{Нижний в центре} % По умолчанию здесь номер страницы

\usepackage{lastpage} % Узнать, сколько всего страниц в документе.

\usepackage{soul} % Модификаторы начертания

\usepackage{hyperref}
\usepackage[usenames,dvipsnames,svgnames,table,rgb]{xcolor}
\hypersetup{				% Гиперссылки
    unicode=true,           % русские буквы в раздела PDF
    pdftitle={Заголовок},   % Заголовок
    pdfauthor={Автор},      % Автор
    pdfsubject={Тема},      % Тема
    pdfcreator={Создатель}, % Создатель
    pdfproducer={Производитель}, % Производитель
    pdfkeywords={keyword1} {key2} {key3}, % Ключевые слова
    colorlinks=true,       	% false: ссылки в рамках; true: цветные ссылки
    linkcolor=red,          % внутренние ссылки
    citecolor=yellow,       % на библиографию
    filecolor=magenta,      % на файлы
    urlcolor=purple         % на URL
}

\usepackage{csquotes} % Инструменты для ссылок


\newcounter{nz}[subsection] % счётчик параграфов

\newcommand{\z}[1]{%        % оформление параграфов

\addtocounter{nz}{1}         
\textbf{\S \arabic{nz}. #1%
}}

\newcounter{ny}             % счётчик для подзаголовков с литературой

\newcommand{\y}[1]{%

\addtocounter{ny}{1}
\vspace{2.5ex} {\textbf {{\Alph{ny}. #1%
}}}}

\newcounter{nx}             % счётчик цитат из Писания внутри списка литературы

\newcommand{\x}[1]{%

\addtocounter{nx}{1}
\textbf{\arabic{nx}. #1%
}}

%%% Дополнительная работа с математикой
\usepackage{amsmath,amsfonts,amssymb,amsthm,mathtools} % AMS

%% Свои команды
\DeclareMathOperator{\sgn}{\mathop{sgn}}

%%% Работа с картинками
%\usepackage{graphicx}  % Для вставки рисунков
%\graphicspath{{images/}{images2/}}  % папки с картинками
%\setlength\fboxsep{3pt} % Отступ рамки \fbox{} от рисунка
%\setlength\fboxrule{1pt} % Толщина линий рамки \fbox{}
%\usepackage{wrapfig} % Обтекание рисунков текстом

%%% Работа с таблицами
%\usepackage{array,tabularx,tabulary,booktabs} % Дополнительная работа с таблицами
%\usepackage{longtable}  % Длинные таблицы
%\usepackage{multirow} % Слияние строк в таблице

%%% Теоремы
\theoremstyle{plain} % Это стиль по умолчанию, его можно не переопределять.
\newtheorem{theorem}{}%[section]
%\newtheorem{proposition}[theorem]{Утверждение}
 
%\theoremstyle{definition} % "Определение"
%\newtheorem{corollary}{Следствие}[theorem]
%\newtheorem{problem}{Задача}[section]
 
%\theoremstyle{remark} % "Примечание"
%\newtheorem*{nonum}{Решение}

%\usepackage{footmisc} % для сносок (тест)


%\usepackage{cite} % Работа с библиографией
%\usepackage[superscript]{cite} % Ссылки в верхних индексах
%\usepackage[nocompress]{cite} % 
%\usepackage{csquotes} % Еще инструменты для ссылок