\section*{Словарь терминов}\addcontentsline{toc}{section}{Словарь терминов}

{\tt Английский перец}, или душистый перец, – сильная пряность, придает блюдам аромат и меняет их вкус. Родина – Антильские острова, но лучшим считается перец, который выращивают на Ямайке.

{\tt Анисовое семя} – плоды зонтичного растения анис обыкновенный. Имеют терпкий и одновременно приятный аромат. Используются как пряность в кондитерском производстве и кулинарии: при выпечке хлеба, засолке овощей, приготовлении напитков.

{\tt Арак} – алкогольный напиток, ароматизированный анисом. Крепость, сырье и технология отличаются в зависимости от региона. Распространен на Ближнем Востоке и в Центральной Азии.

{\tt Базилик} – в переводе с древнегреческого означает «царская трава». Однолетнее растение с разнообразными оттенками аромата – лимонным, мятным, гвоздичным и др. Свежие и сушеные листья базилика применяют в качестве пряной приправы для соусов, овощных маринадов. Широко используется в итальянской и восточной кухне.

{\tt Белорыбица}, или нельма, – одна из самых крупных (достигает 15 кг) сиговых рыб, относящихся к семейству лососевых. Белое нежное мясо белорыбицы хорошо сочетается с различными соусами из свежих трав, цитрусовых и томатов. 

{\tt Белужина} – одна рыбина белуги. Белуга относится к семейству осетровых, считается самой крупной пресноводной рыбой.

{\tt Бешамель} (от фр. bechamel, «белый соус») – классический соус из муки, масла и молока.

{\tt Бланманже} (от фр. blanc – белый, manger – есть) – непрозрачное желе, которое приготовляется из молока, яиц, муки или манной крупы, сахара и желатина. Также добавляют пряности или ароматизатор. В отличие от фруктовых желе, бланманже бывает только белого цвета или оттенков бежевого (если добавляют кофе или какао). В XVIII – начале XIX века в бланманже обязательно использовали орехи: миндаль, фундук, грецкие.

{\tt Бульон куском} – сухой бульон.

{\tt Бурак} – так называют свеклу в южных областях России, на Украине, в Белоруссии.

{\tt Весёлка} – деревянная лопатка для помешивания. Вода померанцевых цветов, или померанцевая вода, – очень ароматная жидкость, применяется как добавка в тесто и в сладости. Побочный продукт производства спирта из цветов померанца, горького апельсина. 

{\tt Вольный дух} – жар в истопленной печи после выгреба углей. 

{\tt Вязига}, или визига, – спинная струна (хорда), проходящая сквозь позвоночник осетровых рыб.

{\tt Галантир} (от франц. galantine) – 1) кушанье из холодной фаршированной дичи; 2) холодная заливная приправа к разным кушаньям, желе (застывший клейкий навар из рыбы, телячьих ножек и т. п.). 

{\tt Гаше} (от франц. hacher, «рубить») – блюдо из жареного или вареного и потом изрубленного мяса.

{\tt Головизна} – 1) голова и часть хребта осетровой рыбы, употребляемые в пищу; 2) вообще рыбьи головы.

{\tt Горчица сарептская}, или русская, – однолетнее растение, семена которого в первую очередь используют для получения горчичного масла, а из жмыха семян получают горчичный порошок для пищевых (горчица «вырви глаз») и медицинских (горчичники) целей. 

{\tt Гуммиарабик}, или аравийская камедь, – твердая прозрачная масса, которую выделяют различные виды акации. В кулинарии используется в качестве эмульгатора (для эмульсий из несмешивающихся жидкостей).

{\tt Драчена}, или дрочёна, – блюдо русской кухни из яиц, замешанных на молоке с крупой, мукой или тертым картофелем.

{\tt Закваска} – вещество, которое вызывает кислое брожение, например дрожжи. 

{\tt Земляная груша} – топинамбур, съедобные клубни (белые, желтые, фиолетовые, красные), по вкусу похожи на репу и капустную кочерыжку.

{\tt Каперсы} – нераспустившиеся бутоны колючего кустарника Capparis spinosa. В сыром виде несъедобны. Каперсы употребляют в пищу маринованными или консервированными в уксусе с солью. Вкус у каперсов пикантный, слегка терпкий, кисловатый, немного горчичный. Они придают оттенок и остроту соусам. Их подают к рыбе или мясу, в салатах или в салатных заправках.

{\tt Каплун} – холощеный, специально откормленный петух и блюдо из него.

{\tt Кардамон} – многолетнее растение семейства имбирных, произрастает в Южной Индии. В качестве специи используются семена, которые имеют сильный аромат и остропряный, слегка сладковатый вкус. Кардамон применяют как в кондитерском деле (для печенья, булочек, пряников, халвы, других сладких блюд), так и в кулинарии (для тушеной птицы, дичи, плова, для соусов).

{\tt Каротель} – сорт моркови с коротким округлым корнем.

{\tt Картофельная мука} – сушеный картофель, размолотый в муку. Картофельную муку иногда ошибочно называют крахмалом.

{\tt Квашня} – 1) деревянная кадка для теста; 2) опара. Забродившее тесто.

{\tt Кервель} – однолетнее травянистое растение семейства зонтичных. Применяется как пряность. Обладает сладковатым анисовым запахом, по вкусу напоминает петрушку. Кервель хорошо сочетается со многими травами: эстрагоном, петрушкой, базиликом, и усиливает их аромат в блюде.

{\tt Кислые щи} – старинный русский сильногазированный безалкогольный напиток из меда и солода. Брожение происходит в запечатанных бутылках, этим технология отличается от приготовления кваса.

{\tt Кишмиш} – виноград с мелкими бессемянными ягодами. 

{\tt Кнель} (от франц. quenelles) – катышки из рубленого мяса или рыбы.

{\tt Коринка} - черный мелкий изюм без косточек. 

{\tt Корчага} – большой, обычно глиняный сосуд. 

{\tt Кострец} – нижняя часть крестца в теле животного.

{\tt Котлета} (от франц. côtele, «ребристый») изначально под котлетой в России понимали кусок мяса с реберной костью. С конца XIX века в русских кулинарных книгах стали появляться «котлеты рубленые», а потом уж котлетами стали называть изделия из фарша.

{\tt Крупитчатая мука (крупчатка)} – отборная мука из специальных сортов твердой пшеницы с высоким содержанием клейковины.

{\tt Лазанки} (от итал. lasagna, «лазанья») – мучное блюдо традиционной белорусской, литовской, польской, русской, украинской кухни. Рецепт появился в XVI веке и был «обработкой» рецепта итальянской пасты.

{\tt Летняя вода} – вода комнатной температуры.

{\tt Майонез} – холодный соус из растительного масла, яичного желтка, уксуса и/или лимонного сока, сахара, поваренной соли, иногда горчицы и других приправ.

{\tt Майоран} – в переводе с арабского значит «несравненный». Пряность, которая придает блюдам особый аромат. Это отличная приправа к баранине, говядине, свинине и жирному гусиному мясу. Им приправляют соусы, картофельные блюда. Хорошо сочетается с тимьяном.

{\tt Малага} – десертное вино крепостью от 15\% до 23\%. Производится из винограда, выращенного в испанской провинции Малага, в Андалусии. По содержанию сахара малага бывает от сухой до сладкой.

{\tt Медовая сыта} – разбавленный в воде мед. По концентрации может быть различной.

{\tt Меренга}, или безе, – пирожное, которое выпекают из взбитых яичных белков и сахара. Используют как для украшения десертов и тортов, так и в качестве основы сладких пирогов (меренговые коржи).

{\tt Муравленый} – покрытый глазурью керамический сосуд.

{\tt Мускатный орех} – сердцевина плодов мускатника, вечнозеленого дерева. Орех (семя) покрыт мясистым присемянником, мягкой кожицей, из которой получают мускатный цвет. У мускатного ореха и мускатного цвета сильный, утонченный аромат и пряно-жгучий вкус, но разных оттенков. Вкус мускатного ореха – сладкий и интенсивный, а у мускатного цвета более горький. Это две разные пряности.

{\tt Мускатный цвет} – см. мускатный орех.

{\tt Мусс} (от фр. mousse, «пена») – сладкий десерт. Делается из трех основных компонентов: ароматической основы (ягодного или фруктового сока, кофе, какао и т. д.), вещества, которое помогает пенообразованию и фиксации пены (желатина, яичных белков) и подсластителя (меда, патоки и т. д.). Также добавляются для вкуса пряности, коньяк, варенье и др.

{\tt Мутовка} – лопаточка, палочка с кружком или спиралью на конце для взбалтывания или взбивания.

{\tt Мякотный кусок} – то же, что мякоть.

{\tt Нашпиговать} – кулинарный прием, применяется в основном для мяса, которое прокалывают или прорезают в нескольких местах, куда вставляют зубчики чеснока, кусочки сала, моркови и т. д. Это делается для повышения жирности мяса или для улучшения его вкуса.

{\tt Огузок} – бедренная часть мясной туши. Самая выгодная часть для чистого крепкого бульона: она мясистая, мягкая и сочная. Также огузок отлично подходит для котлет, биточков, из него получается прекрасное тушеное и отварное мясо. 

{\tt Оправить}, или заправить, – тушке придают более красивую и компактную форму для равномерного прожаривания и удобства нарезки на порционные куски. У птичьих тушек прикрепляют ножки и крылья к туловищу либо с помощью «кармашков» (делают разрезы кожи на брюшке и туда вставляют концы ножек), либо с помощью нитки.

{\tt Осетровый клей} – см. рыбий клей.

{\tt Отцветить} – осветлить, сделать прозрачным бульон, если он получился мутным.

{\tt Папильотка} – бумажный манжет на ножках жареных цыплят, индеек и другой птицы.

{\tt Пастернак} – пряное огородное растение семейства зонтичных, похожее на петрушку. Используется для супов, салатов. Совет: корень пастернака сразу после чистки нужно класть в холодную воду, чтобы он не почернел.

{\tt Пасха} – невареное сладкое блюдо на основе творога. Приготавливается с добавлением яиц, сметаны, сливочного масла, сахара, изюма, цукатов, корицы, эссенций и т. д.

{\tt Патока} – густое сладкое вещество, похожее на свежий жидкий мед. Представляет собой полуфабрикат при заводском производстве сахара и крахмала. Бывает патока белая (крахмальная) и черная (из сахарной свеклы). Используется для пряников и некоторых сортов хлеба (бородинского, карельского, рижского и др.). Добавляет выпечке цвет и особенный вкус.

{\tt Пеклеванный хлеб} – наименование хлеба в XVIII–XIX вв. в России, выпекаемого не из цельной муки, а из муки, прошедшей процесс деления, или пеклевания. 

{\tt Печения} – то же, что выпечка.

{\tt Пикули} – мелкие маринованные овощи. Употребляются как приправа.

{\tt Подболтка} – то, что подбалтывают во что-нибудь для вкуса.

{\tt Пом д’амур} – то же, что томат. Во Франции считали, что томат усиливает и стимулирует половое влечение или половую активность, поэтому назвали его «пом д’амур» – «яблоко любви».

{\tt Померанцы} – разновидность цитрусовых. Зрелые плоды имеют темно-зеленую кожуру, напоминают мелкие лимоны, но по форме круглые. Померанцы ценятся своей цедрой, которая используется в кондитерских изделиях: корки высушивают и перетирают в порошок.

{\tt Потроха} – употребляемые в пищу внутренние органы домашних животных: печень, сердце, почки, сердце, рубец, кишки.

{\tt Приправка}, подправка – то же, что подливка.

{\tt Прованское масло} – оливковое масло высшего сорта.

{\tt Пропускная бумага} – промокательная бумага.

{\tt Пудинг} (англ. pudding – тумба, чугунная болванка, толстое, расплывшееся лицо, а еще глупая голова, набитая всем чем угодно) – английское национальное блюдо. Основными продуктами для пудинга служат белый хлеб, рис или другая крупа, яйца, молоко, масло или жир. Различные мясные или фруктовые компоненты добавляются в зависимости от того, готовится пудинг на второе блюдо или на десерт. Приготовляется на водяной бане.

{\tt Пулярда}, или пулярка, – откормленная жирная курица.

{\tt Рассиропить} – подсластить. 

{\tt Рейнвейн} – сорт виноградного вина, который производят в долине Рейна в Германии. 

{\tt Ренский уксус} – винный уксус, виноградное вино, перешедшее в квасное или кислое брожение. 

{\tt Розовая вода} – вытяжка из лепестков роз, побочный продукт при производстве розового масла.

{\tt Рубец} – самый большой отдел желудка жвачных животных. 

{\tt Русский перец} – стручковый перец, полукустарник с красными плодами-стручками, острыми по вкусу.

{\tt Русское масло} – топленое масло; во многие страны его ввозили под названием русского. 

{\tt Рыбий клей} – желирующее вещество, которое получают из различных органов рыб: плавательных пузырей, чешуи, плавников и т. д., богатых соединительной тканью, содержащей коллаген. Клей наилучшего качества вырабатывается из плавательных пузырей, особенно красной рыбы (осетра, севрюги, белуги и др.). Используется для желирования соков, бульонов, мармеладов и т. д.

{\tt Сабайон}, или савойский соус, – пенистый соус, который приготавливают из взбитых желтков и небольшого количества жидкости. 

{\tt Савой} – савойская капуста, она же итальянская, она же курчавая. Похожа на белокочанную, но имеет нежные гофрированные листья без жестких прожилок. Название происходит от графства Савойя в Италии, где этот вид капусты издавна выращивают. 

{\tt Сальник} – жировая складка в брюшине.

{\tt Сардель}, сарделька – 1) устаревшее название сардины; 2) толстая короткая сосиска. 

{\tt Свекольник} – свекольная ботва.

{\tt Селитра} – при засолке мяса традиционно использовали пищевую калиевую или натриевую селитру, которая служит не только консервантом, но и обеспечивает мясу приятный розоватый цвет, близкий к натуральному. Сейчас в пищевой промышленности селитру стараются не применять: продукты ее разложения, нитраты, считаются ядовитыми.

{\tt Сельдерей} – огородное растение семейства зонтичных. Имеет довольно сильный запах и применяется в качестве ароматной приправы – для супов, овощных и мясных блюд (особенно подходит при приготовлении утки, гуся, баранины), при засолке. В кулинарии используется как листовой сельдерей (сами листья и стебель), так и клубневой.

{\tt Сижок} – сиг, северная промысловая рыба.

{\tt Сладкое мясо} – кулинарное название грудной или зобной железы. По вкусу напоминает свежий хлеб. Дефицитность продукта заключается в том, что когда теленок или ягненок взрослеет, железа эта постепенно атрофируется и у взрослого животного исчезает. 

{\tt Смоленская крупа} – мелкая гречневая крупа величиной с маковое зерно. Применялась для начинок пирогов, для сладких и полусладких каш на молоке.

{\tt Снеток} – небольшая рыбка длиной до 18 см, озерная корюшка.

{\tt Солод} – продукт, который получают из ростков пророщенного зерна (ячменя, ржи, пшеницы, овса, проса). Используется при производстве кваса, пива, спиртных напитков. Продукт богат ферментами (белками, ускоряющими химическую реакцию в организме). 

{\tt Сотерн} – французское белое десертное вино. 

{\tt Ссек} – сорт говядины, мясо от верхней части бедра. 

{\tt Столовое масло} (мызное) – низкосортное масло, промытое меньше сливочного. Хорошее столовое масло в разрезе должно быть однородного цвета (светло-желтого), не должно крошиться и при нажимании не должно выпускать из себя воду. На вкус несоленое и не отдает салом.

{\tt Тельное} – традиционное русское блюдо из рыбы: 1) рыбный фарш (изначальное понятие); 2) пироги и рыба, начиненные рыбным фаршем; 3) зразы из рыбного фарша. 

{\tt Тмин} – растение семейства зонтичных. Используются семена, которые имеют сладковатый аромат и слегка жгучий вкус. Они содержат эфирные масла и применяются не только в кулинарии и кондитерском производстве (маринады, выпечка), но и в народной медицине.

{\tt Точеный} – вырезанный красивыми формочками. 

{\tt Трюфель} – гриб, лакомство для гурманов. Растет под землей – до 30 см от поверхности, поэтому для поиска трюфелей используют специально обученных свиней. Очень дорогой гриб, его цена превышает цену золота.

{\tt Турецкие бобы} – зернобобовая культура. Из зеленых бобов готовят те же блюда, что из стручковой фасоли, только варятся они дольше. Зрелые семена турецких бобов готовят так же, как и зерновую фасоль.

{\tt Фестоны} – зубчатая кайма. 

{\tt Филей} – т о же, что филе: 1) мясо высшего сорта из средней части хребта туши; 2) вообще кусок мяса или рыбы, очищенный от костей.

{\tt Французская водка} – 1) водка, полученная при перегонке виноградного сока; 2) коньяк; 3) продукт перегонки виноградных дрожжей; 4) хлебное вино, переработанное так, что оно теряет специфический хлебный вкус.

{\tt Французский белый хлеб} - 1) французская булка – хлебец из пшеничной муки, продолговатой формы, весом не больше 200 г. В дореволюционной России это был очень популярный сорт хлеба благодаря своему аромату и знаменитой хрустящей корочке. Секрет французских булок заключается не в тесте, а в технологии производства; 2) французский багет – длинное тонкое хлебобулочное изделие весом 250 г.

{\tt Фрикасе} – нарезанное мелкими кусочками жареное или вареное мясо с добавлением приправ.

{\tt Холодник} – холодный суп, приготовленный на свекольном или щавелевом отваре либо каком-то кисломолочном продукте.

{\tt Чабер} – он же чабёр, ароматная трава с перечным вкусом.

{\tt Чухонское} (сметанное, кухонное) масло – коровье масло, которое получали сбиванием сметаны или кислого неснятого молока. Оно шло на нужды кухни; из него делалось также топленое (русское) масло.

{\tt Шалот} – маленький репчатый лук с кисло-сладким вкусом.

{\tt Шафранный} (шафрановый) порошок – пряность, которая придает блюду мягкий золотистый цвет и своеобразный запах. Слово «шафран» происходит от арабского «за-фран», что означает «желтый цвет». Шафран используют для окрашивания и ароматизации кондитерских мучных изделий, блюд из риса, кремов, подливок, а также применяют в производстве сыра, сладостей, ликеров и~т.~д.

{\tt Шиповка} – шипучий алкогольный напиток.

{\tt Шпиговальная игла} – поварской инструмент для шпигования.

{\tt Шумовка} – большая ложка с частыми отверстиями, используется для снятия накипи, для вынимания мяса из бульона.

{\tt Эстрагон}, или тархун, – многолетнее травянистое растение, пряность, используемая в соленьях, при консервировании, приправа к мясным блюдам.

{\tt Язь} – вид рыб из семейства карповых.

{\tt Ячная} (ячневая) крупа – мелкорубленая перловая крупа.