\documentclass[a4paper,12pt]{article}

%%% русский язык
\usepackage[greek,english,russian]{babel}
\usepackage[or]{teubner} %для греческого (см.выше)
\usepackage{cmap}					% поиск в PDF
\usepackage[T2A]{fontenc}			% кодировка
\usepackage[utf8]{inputenc}			% кодировка исходного текста
\usepackage{indentfirst}            % отступ первой красной строки раздела
\frenchspacing

%\usepackage{lastpage} %показывает последнюю страницу

\usepackage{fancyhdr} % Колонтитулы
 	\pagestyle{fancy}
 	\renewcommand{\headrulewidth}{0.5pt}  % Толщина линейки, отчеркивающей верхний колонтитул
 	%\lfoot{Владислав Татаркевич}
 	%\rfoot{История эстетики}
 	%\rhead{Глава}
 	% \chead{Верхний в центре}
 	%\lhead{\copyright Жарова~А.~В., перевод}
 	% \cfoot{Нижний в центре} % По умолчанию здесь номер страницы

\usepackage{lastpage} % Узнать, сколько всего страниц в документе.

\usepackage{soul} % Модификаторы начертания

\usepackage{hyperref}
\usepackage[usenames,dvipsnames,svgnames,table,rgb]{xcolor}
\hypersetup{				% Гиперссылки
    unicode=true,           % русские буквы в раздела PDF
    pdftitle={Заголовок},   % Заголовок
    pdfauthor={Автор},      % Автор
    pdfsubject={Тема},      % Тема
    pdfcreator={Создатель}, % Создатель
    pdfproducer={Производитель}, % Производитель
    pdfkeywords={keyword1} {key2} {key3}, % Ключевые слова
    colorlinks=true,       	% false: ссылки в рамках; true: цветные ссылки
    linkcolor=red,          % внутренние ссылки
    citecolor=yellow,       % на библиографию
    filecolor=magenta,      % на файлы
    urlcolor=purple         % на URL
}

\usepackage{csquotes} % Инструменты для ссылок


\newcounter{nz}[subsection] % счётчик параграфов

\newcommand{\z}[1]{%        % оформление параграфов

\addtocounter{nz}{1}         
\textbf{\S \arabic{nz}. #1%
}}

\newcounter{ny}             % счётчик для подзаголовков с литературой

\newcommand{\y}[1]{%

\addtocounter{ny}{1}
\vspace{2.5ex} {\textbf {{\Alph{ny}. #1%
}}}}

\newcounter{nx}             % счётчик цитат из Писания внутри списка литературы

\newcommand{\x}[1]{%

\addtocounter{nx}{1}
\textbf{\arabic{nx}. #1%
}}

%%% Дополнительная работа с математикой
\usepackage{amsmath,amsfonts,amssymb,amsthm,mathtools} % AMS

%% Свои команды
\DeclareMathOperator{\sgn}{\mathop{sgn}}

%%% Работа с картинками
%\usepackage{graphicx}  % Для вставки рисунков
%\graphicspath{{images/}{images2/}}  % папки с картинками
%\setlength\fboxsep{3pt} % Отступ рамки \fbox{} от рисунка
%\setlength\fboxrule{1pt} % Толщина линий рамки \fbox{}
%\usepackage{wrapfig} % Обтекание рисунков текстом

%%% Работа с таблицами
%\usepackage{array,tabularx,tabulary,booktabs} % Дополнительная работа с таблицами
%\usepackage{longtable}  % Длинные таблицы
%\usepackage{multirow} % Слияние строк в таблице

%%% Теоремы
\theoremstyle{plain} % Это стиль по умолчанию, его можно не переопределять.
\newtheorem{theorem}{}%[section]
%\newtheorem{proposition}[theorem]{Утверждение}
 
%\theoremstyle{definition} % "Определение"
%\newtheorem{corollary}{Следствие}[theorem]
%\newtheorem{problem}{Задача}[section]
 
%\theoremstyle{remark} % "Примечание"
%\newtheorem*{nonum}{Решение}

%\usepackage{footmisc} % для сносок (тест)


%\usepackage{cite} % Работа с библиографией
%\usepackage[superscript]{cite} % Ссылки в верхних индексах
%\usepackage[nocompress]{cite} % 
%\usepackage{csquotes} % Еще инструменты для ссылок

\title{В.Татаркевич, т.2}

\usepackage[backend=biber,bibencoding=utf8,sorting=ynt,maxcitenames=2,style=authoryear]{biblatex}
\addbibresource{bib.bib}

%%% Заголовок
\author{Владислав Татаркевич}
\title{История эстетики \\ 2}
\date{}

\begin{document}

\maketitle

\renewcommand{\thinspace}{\,}

\renewcommand{\thesection}{\alph{section}.}

\renewcommand{\thesubsection}{\arabic{subsection}.}

\renewcommand{\thesubsubsection}{\arabic {subsubsection}.}

% Благодарности

Благодарности тем, кто молодец:

Татаркевич


% Предисловие переводчика

Здесь кратко о том, как переводился текст

\textsl{ноябрь 2016--}

\newpage

\tableofcontents

\newpage

\part{Эстетика раннего Средневековья}
\section{Эстетика Востока}

%первая глава
\subsection{Условия развития}
\z{Христианство.} Ещё до I~в. н.\,э., когда культура эллинизма была в расцвете, а Рим находился на пике силы, начало меняться отношение людей к миру и жизни, которое оторвало их от житейских дел и повернуло к трансцендентному. Во многих сферах рациональное отношение уступило место мистическому, сиюминутные потребности отошли на второй план по сравнению с религиозными. Это новое отношение и новые потребности породили новые религии, секты, обряды, религиозно-философские системы, в целом новый взгляд на мир. А с ним и новую эстетику. Из античных доктрин все материалистические и позитивистские утратили свою привлекательность, а приобрёл её платонизм. Собственным же созданием эпохи была трансцендентная, монистическая, эманационная неоплатоническая система Плотина с его экстатической теорией познания, этикой и эстетикой. 

Однако наиболее плодовитым на последствия было появление христианской религии. Почти через три века после её появления, во время которых её последователи были ещё немногочисленными и не имели влияния на власть и общественность, сохранялись старые формы жизни и мышления: эти столетия ещё относятся к античному периоду. Зато с IV~в., точнее с 313~года, когда согласно эдикта Константина Великого, христианство уже можно было беспрепятственно исповедовать, а особенно с 325~г., когда стало государственной религией, новые формы жизни и мысли взяли верх над старыми: начался новый период в истории <<обитаемого мира>>. 

Христианство опиралось на свою веру, моральный закон, принцип любви, обещание вечной жизни; ему не нужна была наука, философия, в особенности эстетика. <<Любовь к Богу "--- вот истинная философия>>, "--- говорил Иоанн Дамаскин, а Исидор Севильский писал: <<Первая задача науки "--- … Бога, а вторая "--- добродетельность жизни>>. Если же христианство должно было иметь философию, то собственную, не такую, как раньше.

Латинские Отцы Церкви в аскетичной Африке и трезвой римской обстановке желали вовсе отречься от философии. Напротив, Отцы греческой Церкви, действовавшие в Афинах или в Антиохии, в рамках философских традиций, видели, что среди язычников философия также сближается с религией: поняли потребность в философии и возможность использования создания христианской философии в согласии с древними воззрениями. Были предприняты различные попытки: Тертуллиан пробовал переложить христианскую философию наподобие стоической, Григорий Нисский "--- наподобие философии платонической, Ориген "--- неоплатонической; но Церковь не одобрила эти попытки. Однако в IV и V~вв. в письмах Греческих Отцов и Августина уже появилась установленная христианская философия, состоящая из собственной веры и тех элементов античного знания, которые были признаны Церковью. Эстетика не была в этой философии на первом месте, однако присутствовала.

Сложилось так, что первые создатели христианской философии, как Греческие Отцы, так и Августин имели существенные интересы и эстетическую компетенцию. На их эстетике, как и на всей философии, кроме античных доктрин, сложилось \emph{Св.~Писание}. Таким образом, историю эстетики того периода следует начинать с эстетических мотивов, содержащихся в \emph{Св.~Писании}.

\z{Две империи.} 
В том же IV столетии, в котором началась эра царствования христианства и которое установило правила христианской философии и эстетики, произошло другое важное изменение: разделение <<обитаемого мира>> на Восток и Запад.

Различия между восточной и западной частями Римской империи в устройстве и менталитете всегда были велики, но чрезвычайно увеличились в 395~г., когда империя политически была поделена на Восточную и Западную. С тех пор история не только политическая, но и культурная обеих частей империи пошла по своему пути. Западная империя быстро потерпела поражение, Восточная просуществовала тысячу лет. Западная подвергалась преобразованиям, консервативная Восточная сумела остановить развитие. Западная должна была хотя бы частично адаптироваться к обычаям северных завоевателей, Восточная же, находясь на рубеже Восточной Европы, оказалась подвержена влиянию Азии. А прежде всего: на Востоке могли сохраняться формы античной культуры, а на Западе они были уничтожены и забыты. Восток мог жить этими формами античной культуры, сохранять их или развивать, а Запад должен был сам создавать себе культурные формы, утратив античные. Должен был начинать с начала. Он потерял более совершенные формы, зато творил собственные. На Востоке заканчивалась античная история, на Западе начиналась новая. Если называть средневековыми те формы новой культуры, которые создавались на Западе после падения Рима, то на Востоке Средневековья не было; это было исключительно западное явление. В действительности и Восток в то время перешёл к новым формам жизни и культуры, но не начинал с начала, а продолжал античность.

Таким образом историю христианской эстетики, культуры, искусства необходимо развивать в двух направлениях: отдельно на Востоке и на Западе. А начинать надо с Востока, потому что он непосредственно связан с античностью. Здесь на протяжении многих веков греческая мысль и искусство оставались живы, Византия думала и говорила по-гречески, хотя думала также и по-христиански. До VI~в. существовала Платоновская Академия. На Востоке сохранялись многие античные традиции. Византия по замыслу Константина Великого должна была воспринять наследие Рима и действительно стала <<Новым Римом>>. В то же время благодаря историческому и географическому положению Византия была наследницей Греции. Здесь не было недостатка в античных образцах: по приказу императоров со всей империи были собраны и привезены сюда произведения античного искусства. Прямо перед церковью св.~Софии было установлено 427 греческих и римских статуй. В таких условиях очень быстро развивалось христианское искусство; тут начиналась как музыкальное, так и пластическое искусство, тут появлялись первые великие христианские храмы во главе с храмом св.~Софии. И здесь же среди христиан началась эстетическая рефлексия.
%литература после первой главы
\subsection*{Литература}\addcontentsline{toc}{subsubsection}{Литература}
Историки, полагая, что Средние века занимались теологией, самое большее психологией и космологией, долго не предполагали искать в их наследии эстетику, так же долго не было литературы о средневековой эстетике. О ней не хватало информации в общих исследованиях истории философии. А специальные исследования истории эстетики перескакивали период Средневековья. Изложив античную эстетику, переходили непосредственно к изложению эстетики Нового времени.

Действительно, писатели Средневековья не оставили трактатов по эстетике, но в теологических, психологических, космологических трактатах они создавали эстетические основы или делали эстетические выводы, выражали определённое понятие красоты и искусства. Много текстов, интересующих историка эстетики содержат издательства J.~P.~Migne \textit{Patrologia Graeca} (цит. ниже как \textit{P.~G.}) в 161 томах и \textit{Patrologia Latina} (цит. \textit{P.~L.}) в 221 томах, а также позднейшие, преимущественно лучшие издания средневековых писателей. Часть их произведений является ещё не изданными рукописями.

Первыми работами по средневековой эстетике были монографические исследования конца XIX~в. об Августине, потом о Фоме. Их было мало и они охватывали незначительную часть темы. В то время как после Второй мировой войны сразу появилось исследование всего наследия Средневековья: Edgar  D~e~~B~r~u~y~n~e,  \textit{Études d’Esthétique Médiévale}, в 3-х томах, изд.~Гентского университета, 1946. Благодаря работе одного человека средневековые материалы по истории эстетики стали собраны полней, чем античные материалы. Они более специально разработаны, но не объединены ещё из сотен монографических работ. На материале, который собрал де Брюйне, основана в немалой степени данная работа: эти материалы требовали скорей сокращений и отбора, потому что вместе с важными для эстетики текстами включают в себя множество текстов несущественных. Кроме издания этих Работ, послуживших источником, де Брюйне дважды систематически изложил средневековую эстетику: по-французски: \textit{Esthétique du Moyen Age}, Лёвен 1947 и по-фламандски: \textit{Geschiedenis van de Aesthetica de Middeleeuwen}, Антверпен 1951--1955.

Материалы, собранные де Брюйне не охватывают эстетики восточного христианства, а эстетику Запада начинают после Августина. О Боэции де Брюйне пишет в т.~I, стр.~3, о Кассиодоре I, 35, об Исидоре I, 74, об эстетике эпохи Каролингов I, 165, о средневековой поэтике I, 216 и II, 3, о средневековой теории музыки I, 306 и II, 108, о теории пластического искусства I, 243, II, 69 и II, 371, об эстетике мистиков III, 30, об эстетике викторианцев II, 203, о Гильоме~XII, Гильоме Оссерском и о Сумме Александра Гэльского III, 72, о Роберте Гроссетесте III, 121, о Бонавентуре III, 189, об Альберте Великом III, 153, о Фоме Аквинском III, 278, о Витело III, 239.

Де Брюйне пишет в предисловии, что он намеревался дать \emph{un recueil de textes devant servir à l’histoire de l’esthétique médiévale}, но отказался от этого намерения. Тексты он приводит частью в исследовании, частью в сносках, преимущественно в оригинале, иногда с переводом, иногда только по-французски. Кроме этих  работ собраний источников по средневековой эстетике нет. Таким образом, данная работа берёт на себя задачу составления сборника текстов, представляющихся наиболее значимыми, по аналогии с античной эстетикой. Сборник не является полным, но некоторые мысли из области эстетики повторялись средневековыми авторами так часто, что полное собрание перестанет быть полезным из-за своего однообразия; более важным представляется выбор типовых текстов.

Единственный до сих пор большой сборник текстов по средневековой эстетике в итальянском переводе в \textit{Grande Antologia Filosofica}, т.~V, 1954: R.~~M~o~n~t~a~n~o, \textit{L’estetica nel pensiero cristiano}, стр.~207--310.

После работ де Брюйне лучшее синтетическое исследование средневековой эстетики имеет итальянская литература в сборнике \textit{Momenti e problemi di storia dell’estetica}, т.~I, 1959, а именно: Q.~~C~a~t~a~u~d~e~l~l~a,  \textit{Estetica cristiana}, стр.~81--114, а также U.~~E~c~o,  \textit{Sviluppo dell’estetica medievale}, стр.~115--229.

Эта публикация содержит также наиболее полное собрание литературы по предмету исследования (стр. 113--114 и 217--229). Следовало её дополнить главным образом некоторыми работами по истории литературы, музыки и пластических искусств, включающими в себя общие рассмотрение эстетической природы. Вся монографическая литература, касающаяся средневековой эстетики, сильно ограничена, имеет большие пробелы. В данной \emph{Истории} важные позиции приведены в ссылках, в особенности те, которые связаны с эстетикой \emph{Св. Писания} (на стр.~\pageref{sec:pismo_sw}), с эстетикой Отцов Церкви (на стр.~\pageref{sec:greccy}), с византинистикой (на стр.~\pageref{sec:bizantynistyka}), с эстетикой Августина (на стр.~\pageref{subsec:sw_august}, 56, 58, 64, 65), Фомы Аквинского (на стр. 268), с эстетикой пластического искусства (на стр. 157, 159, 160, 162, 163, 167, 170, 184), музыки (на стр. 140), поэзии (на стр. 128, 130).
%вторая глава
\subsection{Эстетика Святого Писания}\label{sec:pismo_sw}

Те же люди, положившие начало всей христианской философии, стояли у истоков христианской эстетики: с одной стороны это были Греческие Отцы, особенно св.~Василий, а с другой стороны "--- Латинские Отцы во главе со св.~Августином. Первые имеют греческие истоки, вторые "--- римские: и те, и другие были знакомы с античным пониманием красоты и искусства и обращались к нему. Это был первый источник их эстетических воззрений "--- но вторым была их собственная христианская идеология, содержащаяся в \emph{Св.~Писании}. Хотя оно служило не эстетическим целям, в нём были обнаружены идеи и на эту тему. Особенно в \emph{Ветхом Завете}.

Слово <<красивый>> (\textgreek{kal'os}) неоднократно появляется в Септуагинте, греческом переводе \emph{Св.~Писания}. В ней были ставились и обсуждались некоторые эстетические вопросы. Из книг Ветхого Завета больше всего таких вопросов поднимают две книги, каждая из которых имеет совершенно особый характер: \emph{Книга Бытия} и \emph{Книга Премудрости}. На переднем плане красота есть также в \emph{Песни Песней}. В \emph{Екклесиасте} и \emph{Книге притч Соломоновых} о ней говорится уже реже.

\z{Книга Бытия.} \emph{Книга Бытия} уже в первом разделе содержит сообщение, которое имеет большое значение для эстетики, потому что касается красоты мира. Там говорится о том, что Бог, осматривая созданный им самим мир, оценивает свою работу. В \emph{Книге Бытия} говорится: <<И увидел Бог всё, что Он создал, и вот, хорошо весьма>>\textsuperscript{\ref{cyt:byt1:31}}. Этот оборот повторяется в \emph{Книге} не единожды, а многократно (Книга Бытия). В нём можно проследить две идеи: во-первых, уверенность в том, что мир прекрасен (уверенность в \emph{панкалии}, как говорится по-гречески), во-вторых, уверенность, что прекрасен потому, что является сознательным творением мыслящей сущности, подобно произведению искусства.

Эти мысли о прекрасном \emph{Книга Бытия} несомненно содержит в своём греческом переводе. Но похоже, что их не было в оригинале, и что они были привнесены при переводе, потому что смысл еврейского оригинала, по мнению исследователей, был другим. Греческим словом \textgreek{kal'os}, или <<прекрасный>> переводчики \emph{Септуагинты}, александрийские учёные-евреи, в III~в. до~н.\,э. перевели прилагательное в более широком значении, означающем внутренние достоинства (особенно моральные: храбрый, полезный, хороший), а также достоинства внешние, но не только эстетические. Собственный смысл слов \emph{Книги}, в которых Бог оценивает свою работу, был таков: работа завершилась у~с~п~е~ш~н~о. Эти слова содержали общую положительную оценку мира, а не только эстетическую; специальной эстетической оценки в них не было Это согласуется с общим подходом \emph{Ветхого Завета} и с тем, что красота почти не играла никакой роли в культе и библейской религиозности.

У переводчиков, помимо этого, было основание для употребления слова \textgreek{kal'on}, которое тоже имело широкий смысл, множество оттенков, означало не только эстетическую красоту, но также моральную и в целом всё, что заслуживает признания и пробуждает удовольствие. Возможно, что его использовали не имея в виду эстетической красоты, и что только впоследствии слово обрело такой смысл. Но может быть и так, что уже сами переводчики так его интерпретировали: потому что в III~в. до~н.\,э. в Александрии интеллектуальная культура была греческой и евреи тоже были подвержены греческому влиянию, которое склоняло к другому, нежели чисто моралистическому, отношению к миру.

Так или иначе, умышленно или нет, переводя библейскую идею о том, что мир удался, словом \textgreek{kal'os}, переводчики Септуагинты привнесли в Библию греческую идею о красоте мира. Даже если это не было целью их перевода, это стало его результатом. Однажды привнесённая, эта идея действовала дальше. Не прошла в латинский перевод Писания , в Вульгату, которая \textgreek{kal'on} перевела как {\sl bonum}, а не {\sl pulchrum}. Однако осталась в христианстве, в культуре Средневековья и Нового времени.

Эта христианская эстетика, провозглашающая красоту мира, хотя основывалась на Ветхом Завете, имела иной источник; нельзя даже утверждать, что она имела два источника, греческий и библейский, потому что целиком была греческой. То, что кажется библейской эстетикой, было греческого происхождения, попало в Библию под греческим влиянием, посредством перевода на греческий.

Идея о красоте творения в том виде, в каком  она появляется в Книге Бытия, возвращается в Книге Премудрости (13, 7 и 13, 5), Екклесиасте (43, 9 и 39, 16)\textsuperscript{\ref{cyt:tyk39:21}}, где деяния Иеговы в природе и истории называются по-гречески \textgreek{kal'a}. Та же идея Септуагинты существует в Екклесиасте (3, 11)\textsuperscript{\ref{cyt:ta3:11}}, а также в Псалме 25, 8: <<Господи, я влюбился в красоту дома Твоего>>\textsuperscript{\ref{cyt:ps25:8}}. Также несколько иными словами в Псалме 95, 5\textsuperscript{\ref{cyt:ps95:5}}; где используется слово \textgreek{<wra\~ios}, имеющее более эстетической значение, чем \textgreek{kal'os}. Во всех этих местах \emph{Св.~Писания} можно разглядеть отзвук эстетических воззрений эллинизма.

Все эти книги Ветхого Завета появились в эллинистический период: Екклесиаст появился в III~в. до~н.\,э., \emph{Книга притч Соломоновых} в начале II~в., а \emph{Книга Премудрости}, даже только в I~в. до~н.\,э., а значит, в период, когда евреи-теологи, такие как Филон Александрийский, хорошо знали эллинистическую философию.

\z{Книга Премудрости.} \emph{Книга Премудрости} "--- это та книга \emph{Ветхого Завета}, в которой чаще всего упоминается красота. Она проповедовала красоту творения, видела в этой красоте доказательство существования и деятельности Бога; через величие и красоту творения познаётся творец, вызвавший их к жизни (13, 5)\textsuperscript{\ref{cyt:prem13:5}}. Говорила о  красоте не только божественных творений, но и человеческих, не только природы, но и искусства, очарование которых так велико, что <<люди приписывают им божественность>>.

Но кроме того эта книга привнесла совершенно иной, не религиозный, а философский мотив, чисто греческий, пифагорейско-платоновский.  Она утверждала, что именно Бог устроил <<всё в соответствии с мерой, числом и весом>> {\sl omnia in mensura et numero et pondere} (11, 21)\textsuperscript{\ref{cyt:prem11:21}}. Провозгласила математическо-эстетическую теорию. Такая теория в религиозной книге была уже особым проявлением греческого влияния: в данном случае не только на перевод, но и на саму книгу. Тот факт, что эта теория оказалась на страницах \emph{Св.~Писания}, для средневековой эстетики имел огромное значение, авторитет \emph{Писания} позволил учёным утвердить её и вызвал это неожиданное явление, что математическая теория стала одной из главных эстетических теорий религиозного периода. Эта идея не единожды появляется в \emph{Ветхом Завете}: также в \emph{Книге притч Соломоновых} сказано, что Бог творение своё посчитал и измерил, {\sl denumeravit et mensus est} (1, 9)\textsuperscript{\ref{cyt:tyk1:9}}.

\z{Екклесиаст и Песнь песней.}
В то время как благодаря грекам в эстетику \emph{Св.~Писания} вошли оптимистические и математические мотивы, то от самих израильтян в Ветхом Завете появилось нечто совсем другое: собственное б~е~з~р~а~з~л~и~ч~н~о~е отношение к прекрасному и в целом к внешнему виду вещей. Рассказывая о зданиях, израильтяне описывали, как те были построены, но никогда "--- как выглядели; о людях "--- Иосифе, Давиде или Авессаломе "--- правдиво писали, что они прекрасны, но не описывали их красоты. Не проявляли интереса к внешнему виду предметов и людей, их мысль не останавливалась на нём, как если бы ускользал от их внимания; если говоря о людях, обращали внимание на особенности внешности, то только на те, которые выражали внутренние переживания.

От безразличного отношения было недалеко до нежелания. Книга притч Соломоновых выразила убеждённость в тщетности красоты: {\sl Fallax gratia et vana est pulchritudo} (31, 30)\textsuperscript{\ref{cyt:prit31:30}}. Можно было бы и здесь усматривать греческий мотив, появившийся из скептических кругов греческой философии; но презрительного отношения к красоте было в \emph{Библии} больше, чем в греческой философии. Примечательно, что о чувственной, зримой, эстетической красоте \emph{Св. Писание} говорит, описывая древо познания добра и зла, которое \emph{Вульгата} называет {\sl pulchrum oculis aspectuque delectabile}, <<прекрасным для очей и приятным для взора>>: а значит в случае, когда шла речь о чём-то угрожающем, об источнике человеческих бед.

Это негативное отношение \emph{Ветхого Завета} к красоте не нашёл широкого резонанса среди христиан; не помешал тому, что среди них были также те, кто преклонялся перед красотой, видел в ней благо, данное Богом, свидетельство Его совершенства. Две противоположных позиции по отношению к красоте "--- тщетность красоты и красота как признак божественности "--- будут постоянно на протяжении веков проявляться в христианской эстетике.

Другой особенностью отношения израильтян к внешнему виду вещей было то, что его трактовали как с~и~м~в~о~л. Они были уверены, что видимое не является важным само по себе, но только как знак невидимого. О красоте человека являются его свойства в~н~у~т~р~е~н~н~и~е, проявляющиеся в его облике. Если израильтяне вырезали или рисовали своих пророков, жертву Исаака или Моисея в горящем кусте (они делали это редко, но делали, как показывают раскопки III\,в. в Dura Europos на Евфрате), то для того, чтобы представить деятельность Бога; язычники рисовали и вырезали своих богов, они же "--- символы и деяния своего Бога. Христиане частично переняли их воззрения, но переняли также и античные воззрения. И благодаря этому их отношение к красоте было двояким: непосредственный и символичный, и оба проявлялись в их эстетике.

Ещё одна особенность понимания израильтянами красоты нашла выражение в \emph{Песни Песней}, в содержащемся в ней описании красоты невесты. Там указаны двоякие черты, описывающие красоту: с одной стороны такие, как моральная чистота и недоступность, которые сравниваются с башней и крепостью: они являются внутренними достоинствами, проявляющимися внешне. Но с другой стороны, невеста имеет достоинства, составляющие её очарование, которые сравниваются с цветами, украшениями, с тем, что приятно на вкус и запах, со сладостью вина, с благовониями Ливана, шафраном, алоэ, с источником пресной воды. 

Как первые, так и вторые черты изображают иное, чем у греков, понимание прекрасного.

\z{Еврейское и греческое понимание прекрасного.}
1.~Вещи имели для греков н~е~п~о~с~р~е~д~с~т~в~е~н~н~у~ю красоту, а в \emph{Ветхом Завете} "--- опосредованную, символическую. 2.~Для греков о прекрасном являлись о~с~о~б~е~н~н~о~с~т~и вещей, тут "--- их действия, в~п~е~ч~а~т~л~е~н~и~я, которые они производят. 3.~Для греков прекрасное было, как правило, в~и~з~у~а~л~ь~н~о~е, тут "--- в той же, если не в большей степени, было предметом других смыслов, знаков, запахов, звуков; было одновременно z ponętą zmysłową, которая в других смыслах является не менее, если не более сильной. А также 4.~для греков красота была гармонией, то есть гармоничным р~а~с~п~о~л~о~ж~е~н~и~е~м составляющих частей, тут "--- было свойством отдельных элементов; для греков лежало в сочетании предметов, тут "--- в их разобщённости, красивым было прежде всего то, что не смешано, чисто. Одноцветные и светящиеся солнце и месяц были для израильтян \emph{Ветхого Завета} красивей, чем какое-либо сочетание цветов, и аналогично дело обстояло с музыкой. 5.~Таким образом, когда у греков красивой являлась ф~о~р~м~а, то тут "--- и~н~т~е~н~с~и~в~н~о~с~т~ь свойства, цвета, света, запаха, звуков. Красота содержалась для израильтян в том, что живёт и творит, we wdzięku i sile, а не совершенной пропорции, не в форме. <<Величаюшую красоту израильтяне находили в бесформенном и пугающем огне и животворящем свете>>. 6.~В то время, как греки были чувствительны к ~ц~в~е~т~у вне формы, израильтяне скорей к ~с~в~е~т~у; были более чувствительны к яркости света, чем к насыщенности цвета. И отношение к цвету имели другое: в то время, как можно было предполагать, что для греков прекраснейшим цветом был голубой, цвет неба и глаз Афины, то израильтяне не имели для него даже, как утверждают филологи, непосредственного названия; для них красивым был красный цвет. Далее: 7.~Красота классического греческого периода было с~т~а~т~и~ч~н~ы~м, было красотой покоя и равновесия, для израильтян же красота с самого начала была динамичной, была красота движения, жизни, действия. 8.~У греков основной идеей было, что красота есть в природе; у израильтян красота природы играла незначительную  роль. 9.~Греки вырезали своих богов, а у израильтян существовал запрет на изображение Бога. Поскольку понимали Его не образно, красота в прямом значении слова не могла быть Его свойством. Действительно, сказано в \emph{Писании}, что Бог создал человека <<по образу и подобию своему>>, но это {\sl imago Dei} понималось не как воссоздание телесного внешнего вида Бога, но как телесный образ бестелесного Бога, это {\sl imago} было тут формой откровения, а не подобия.

Бога \emph{Ветхого Завета} отличали самые высокие атрибуты, среди которых было величие или великолепие, но не было --- красоты. И всё же хотя и не в \emph{Книгах Моисея} \emph{Ветхого Завета}, а в \emph{Песни Песней} есть такое предложение: {\sl ostende mihi faciem tuam… facies tua speciosa}\textsuperscript{\ref{cyt:prit2:14}}. С господствующим среди израильтян взглядом это предложение удаётся согласовать, только если принять, что слово {\sl speciosus} "--- <<красивый>> использовано в другом значении, а именно в таком, которое могло бы относиться к божеству: красивый, как пробуждающий не чувственную, но исключительно умственную привлекательность. В переносном значении говорил также о красоте Бога мыслитель, близко с \emph{Ветхим Заветом} связанный, Филон Александрийский; видение того, что не сотворённое и божественное "--- писал он "--- есть лучшее из хорошего и прекраснейшее из прекрасного\textsuperscript{\ref{cyt:gaj5}}. Это сублимированное понятие прекрасного закрепилось в христианской эстетике.

Понимание прекрасного, которому даёт определение \emph{Ветхий Завет}, имело, скорее всего, не единственный источник: появилось из условий жизни израильтян, из их монотеистической религии, а в особенности из запретов, которые эта религия привносила.

\z{Запрет на изображения.}
Моисей запретил изображать Бога; даже больше, даже каких-либо живых существ. В \emph{Книгах Моисеевых} этот запрет был сформулирован максимально решительным образом и повторяется не менее 8 раз (Исх. 20, Исх. 20, 23; Исх. 34, 17; Лев. 26, 1; Втор. 4, 15; Втор. 4, 23; Втор. 5, 8; Втор. 27, 15). Шесть раз сказано, что нельзя robić bogów, один раз --- что нельзя создавать никаких скульптурных подобий и ничего <<na kształt mężczyzny lub niewiasty>>, четырежды "--- не ваять каких-либо живых существ, никаких подобий тому, что есть наверху в небе и что внизу на земле, ни тому, что в воде под землёй\textsuperscript{\ref{cyt:ish20:4}}. Запрещались <<подобия>> вообще, но особенно скульптуры, литые, вырезанные или кованые; один раз упоминаются оба вида, запрещая скульптуры как литые, так и вырезанные (Втор. 27, 15)\textsuperscript{\ref{cyt:wtor27:15}}.

Смысл этих запретов несомненный: имели религиозный характер, были введены для того, чтобы предотвратить идолопоклонство. Их радикализм был особенный: он включал в себя все живые существа. А также результат этих запретов: их скрупулёзно соблюдали на протяжении столетий. Они привели к тому, что у израильтян не было ни скульпторов, ни художников, что они по сути перестали заниматься изобразительным искусством. Дальнейшим их результатом было то, что эстетические потребности народа уменьшились. А если и выражались, то не в изящные формы, но в богатстве материи. \emph{Иезекиль} пишет (28, 13): <<твои одежды были украшены всякими драгоценными камнями; рубин, топаз и алмаз, хризолит, оникс, яспис, сапфир, карбункул и изумруд и золото>>. Драгоценность и великолепие были для израильтян самой большой красотой.

\z{Античное наследие.}
Подводя итог, следует сказать следующее: т~р~и мотива, связанные с эстетикой, имели в \emph{Ветхом Завете} большое значение; во-первых, красота вселенной, во-вторых, происхождение красоты из <<меры, числа и веса>>, в-третьих, тщетность и даже небезопасность красоты. Итак, все три были известны грекам: первый был эллинистическим мотивом п~а~н~к~а~л~и~и, второй "--- пифагорейским мотивом м~е~р~ы, третий "--- мотивом к~и~н~и~ч~е~с~к~и~м. Два первых, по всей видимости, попали в \emph{Св.~Писание} от греков, и только до третьего автор \emph{Екклесиаста} додумался сам.

Мотив меры некоторые историки называют <<мотивом мудрости>>, или мотивом \emph{Книги Премудрости}, но она взяла его несомненно из античности, не придумала сама. Скорей мотив \emph{панкалии}, хоть тоже начавшийся в античности, может быть тесней связан с \emph{Библией}, потому что получил в ней значение, которого не имел у греков, и мог бы с большей вероятностью  выступать в качестве <<библейского>> мотива.

Эстетика христиан черпала из обоих источников: из \emph{Ветхого Завета} и греческих авторов. То, что \emph{Ветхий Завет} "--- в некоторых своих книгах, а особенно в переводе \emph{Септуагинты} "--- сам черпал из греков, облегчало объединение обоих источников. Однако двойственность, напряжённость и противоречия остались. Эстетика христиан знала красоту символическую, как непосредственную, красоту света, как красоту гармонии, красоту жизни, как красоту покоя, видела в красоте {\sl vana pulchritudo}, это снова одно из наивысших совершенств мира. Тертуллиан выступал в поддержку сохранения запрета изготовления подобий (однако с другим доводом: чтобы избежать лжи, которая есть в каждом воспроизведении), но большая часть христиан пошла вслед за греками, занималась искусством и представляла в них не только творение, но и самого Творца. Двойственность источников христианства отразилась главным образом в практике, во вкусе, удовольствиях, произведениях искусства, и меньше в теории, в научных обобщениях, потому что в них христиане следовали за античностью.

Ранние христиане жили в свете эллинистическом и если они ставили перед собой эстетические вопросы, то руководствовались при этом эллинистическими понятиями; многие из тех понятий в их эпохе были уже действующими понятиями. Более образованные знали также теории греческих учёных, как распространённый эклектиками взгляд, что красота "--- в расположении частей, а также как новая теория Плотина, что красота "--- в свете и блеске.

Но перенятые из античности эстетические взгляды получили у христиан другое основание, наполнились другим значением. Это произошло благодаря их религиозному отношению, всех значений, относящихся к Богу, и моральному отношению, подчиняющему все задачи человека морали. Мир прекрасен "--- потому что его создал Бог. Мир измерен и учтён "--- потому что это сделал Бог. Красота тщетна "--- по отношению к вечности и моральных задач, стоящих перед людьми. С учётом этих предположений переход от эстетики античной к христианской "--- хотя христиане переняли основные мысли греков и римлян "--- создало новые мотивы: не подробными идеями, но посредством введения нового взгляда на мир.

\z{Евангелие.}
Мировоззрение христиан опиралось прежде всего на \emph{Новый Завет}. Но он содержал ещё меньше эстетических мотивов, чем \emph{Ветхий}; можно даже утверждать, что их там вовсе не было. В действительности слово <<прекрасный>> (\textgreek{kal'os}) встречается там неоднократно: \emph{Евангелие от Матфея} говорит, что дерево дало прекрасные плоды (7, 17; 12, 33), что сеятель сеет прекрасное зерно (13, 27; 37, 38), а особенно "--- прекрасными являются действия (5,16). Впрочем, красота, о которой тут говорится, не является эстетической красотой, но всегда и исключительно моральной; является красотой и добром в христианском понимании, то есть в значении совершенного действия любви и веры. Когда евангелист св.~Иоанн говорит о \textgreek{<o poim`hn <o kal'os} (10, 11 и 14), то можно это переводить только как <<добрый пастырь>>, а не <<красивый>>. Так же и в ранней христианской литературе \textgreek{kal`os n'omos} употребляется в смысле <<доброго закона>>, а  \textgreek{kal`os di'akonos} в смысле <<доброго священника>>.

Из всех разновидностей прекрасного, которые знал эллинизм, \emph{Евангелие} высоко ставило только одну: красоту в моральном значении. Красота в чисто эстетическом смысле, красота внешнего вида или формы, не была для него существенной. Но однако её не обошло стороной, не пренебрегло красивым <<украшением>> предметов; в \emph{Нагорной проповеди}  (Мф., 6, 28—29) сказано: <<Посмотрите на полевые лилии, как они растут: ни трудятся, ни прядут; но говорю вам, что и Соломон во всей славе своей не одевался так, как всякая из них>>\textsuperscript{\ref{cyt:mat6:28}}. Мир телесный и его красота имеет свой вес, потому что как говорит св.~Павел, через него <<wiekuista moc i bóstwo… dla umysłu widzialnymi się stały>>\textsuperscript{\ref{cyt:rim1:20}}.

Ранние христиане не находили в \emph{Евангелии} особенных эстетических утверждений; однако находили в нём подсказку, какую позицию занимать в отношении каждой сферы жизни, а значит также в отношении красоты и искусства. Эта позиция опиралась на убеждение о превосходстве вечных благ над временными, духовных над телесными, моральных над всеми остальными. Не было в \emph{Новом Завете} эстетических теорий, но был пример, какие эстетические теории христиане могут считать своими. И немного времени прошло, и христианские мыслители воспользовались этим примером.
%литература после второй главы

\y{Тексты из Св.~Писания}\addcontentsline{toc}{subsubsection}{\refname}


\begin{theorem}\label{cyt:byt1:31}%1
\textnormal{\textgreek{Ka`i e\~>iden <o je`os t`a pant\'a, <'osa >epo\'ihsen ka`i >ido`u kal`a li\'an.}
(Viditque Deus cuncta, quae fecerat, et erant valde bona).
Красота мира:
<<И увидел Бог всё, что Он создал, и вот, хорошо весьма>>.
Бытие, 1:31}
\end{theorem}

\begin{theorem}\label{cyt:tyk39:21}%2
\textnormal {\textgreek{T`a >'erga kur'iou p'anta <'oti kal`a sf'odra.}
(Opera domini universa bona valde).
<<все дела Господа весьма благотворны>>
EKLEZJASTYK, 39:21 ??? % русский перевод взят из Сираха
}
\end{theorem}

\begin{theorem}\label{cyt:ta3:11}%3
\textnormal {\textgreek{s'un t`a p'anta >epo'ihsen kal`a >en kair\~w| a>uto\~u.}
(Cuncta fecit bona in tempore suo).
<<Всё соделал Он прекрасным в своё время>>.
Екклесиаст, 3:11}
\end{theorem}

\begin{theorem}\label{cyt:ps25:8}%4
\textnormal{\textgreek{k'urie, >hg'aphsa e>upr'epeian o>'ikou sou.}
(Domine, dilexi decorem domus tuae).
<<Господи! возлюбил я обитель дома Твоего>>. 
Псалом, 25:8}
\end{theorem}

\begin{theorem}\label{cyt:ps95:5}%5
\textnormal{\textgreek{>exomol'oghsis ka`i <wrai'oths >en'wpion a>uto\~u.}
(Majestas et decor praecedunt eum)
<<Ибо все боги народов — идолы, а Господь небеса сотворил>>.
Псалом, 95:5}
\end{theorem}

\begin{theorem}\label{cyt:prem13:5}%6
\textnormal{\textgreek{>Ek g`ar meg'ejous ka`i kallon\~hs ktism'atwn >anal'ogws <o genesiourg`os a>ut\~wn jewre\~itai.}
(A magnitudine enim speciei et creaturae cognoscibiliter poterit creator horum videri).
Красота мира указывает Творца
<<ибо от величия красоты созданий сравнительно познается Виновник бытия их>>.
Книга Премудрости, 13:5}
\end{theorem}
	
\begin{theorem}\label{cyt:prem11:21}%7
\textnormal{\textgreek{p'anta m'etrw| ka`i >arijm\~w| ka`i stajm\~w| di'etaxas.}
(Omnia in mensura et numero et pondere disposuisti).
ŚWIAT ZAWDZIĘCZA SWE PIĘKNO MIERZE, LICZBIE I WADZE
<<Ты всё расположил мерою, числом и весом>>.
Книга Премудрости, 11:21}
\end{theorem}

\begin{theorem}\label{cyt:tyk1:9}%8
\textnormal{\textgreek{kuri'os a>ut`os >'ektisen a>ut'hn ka`i e\~>ide kai >exer'ijmhsen a>ut'hn.}
(Ille creavit illam in Spiritu sancto et vidit, et dinumeravit, et mensus est).
<<Он произвёл её (мудрость) и видел и измерил её и излил её на все дела Свои>>. 
EKLEZJASTYK, 1:9 % русский перевод взят из Сираха
}
\end{theorem}

\begin{theorem}\label{cyt:prit31:30}%9
\textnormal{\textgreek{yeude\~is >areske'iai, ka`i m'ataion k'allos.}
(Fallax gratia et vana est pulchritudo)
Ничтожность красоты
<<Миловидность обманчива и красота суетна>>. 
Книга притчей Соломона, 31:30}
\end{theorem}

\begin{theorem}\label{cyt:prit2:14}%10
\textnormal{{Ostende mihi faciem tuam et auditum fac mihi vocem tuam, quoniam vox tua s~u~a~v~i~s est mihi et facies tua s~p~e~c~i~o~s~a.}
Красота Бога % тут про женщину, вообще-то
<<покажи мне лице твое, дай мне услышать голос твой, потому что голос твой сладок и лице твое приятно>>.
Песнь песней, 2:14}
\end{theorem}

\begin{theorem}\label{cyt:gaj5}%11
\textnormal{\textgreek{T`o >ag'enhton ka`i je\~ion <or\~an \ldots t`o kre'itton m`en agajo\~u, k'allion d`e kalo\~u.}
<<и это узренье Бога я ставлю выше всех прочих вещей, для каждого (из нас) и вместе для всех людей>>.
Филон Александрийский, <<О посольстве к Гаю>>, 5.}
\end{theorem}

\begin{theorem}\label{cyt:ish20:4}%12
\textnormal{\textgreek{O>u poie'hseis seaut\~w| e>'idwlon o>ud`e pant`os <omo'iwma <'osa >en t\~w| o>uram\~w| >'anw kai <'osa >en t\~h| g\~h| k'atw kai <'osa >en to\~is <'udasin <upok'atw t\~hs g\~hs.}
(Non facies tibi sculptile neque omnem similitudinem quae est in coelo desuper et quae in terra deorsum nec eorum quae sunt in aquis sub terra).
Запрет на изображения
<<Не делай себе кумира и никакого изображения того, что на небе вверху, и что на земле внизу, и что в воде ниже земли>>. 
Исход, 20:4}
\end{theorem}

\begin{theorem}\label{cyt:wtor27:15}%13
\textnormal{\textgreek{>epikat'aratos >'anjropos <'ostis poi'hsei glupt`on ka`i qwneut'on, bd'elugma kur'iw|, >'ergon qeir\~wn teqnit\~wn.}
(Maledictus homo, qui facit sculptibile et conflatibile, abominationem Domini, opus manuum artificum).
<<проклят, кто сделает изваянный или литый кумир, мерзость пред Господом, произведение рук художника>>.
Второзаконие, 27:15}
\end{theorem}

\begin{theorem}\label{cyt:mat6:28}%14
\textnormal{Considerate lilia agri quomodo crescunt: non laborant neque nent. Dico autem vobis quoniam nec Salomon in omni gloria sua coopertus est sicut unum ex istis.
Красота природы
<<Посмотрите на полевые лилии, как они растут: ни трудятся, ни прядут; но говорю вам, что и Соломон во всей славе своей не одевался так, как всякая из них>>.
Евангелие от Матфея, 6:28-29 % в оригинале (5:28-29), но там не про лилии, а про вожделение
}
\end{theorem}

\begin{theorem}\label{cyt:rim1:20}%15
\textnormal{Invisibilia enim ipsius a creatura mundi, per ea quae facta sunt, intellecta conspiciuntur.
Бог проявляется в Творении.
<<Ибо невидимое Его, вечная сила Его и Божество, от создания мира через рассматривание творений видимы, так что они безответны>>.
Послание апостола Павла к римлянам, 1:20}
\end{theorem}
%третья глава

\subsection{Эстетика Греческих Отцов Церкви}\label{sec:greccy}

\z{Отцы Церкви.} 

\z{Василий Великий.}

\y{Тексты Греческих Отцов Церкви}\addcontentsline{toc}{subsubsection}{Тексты Греческих Отцов Церкви}

\subsection{Эстетика Псевдо-Дионисия Ареопагита}\label{sec:pseudo_dionizy}

\y{Тексты Псевдо-Дионисия Ареопагита}\addcontentsline{toc}{subsubsection}{Тексты Псевдо-Дионисия Ареопагита}

\subsection{Византийская эстетика}\label{sec:bizantynistyka}

\y{Тексты византийских теологов}\addcontentsline{toc}{subsubsection}{Тексты византийских теологов}

\newpage
\section{Эстетика Запада}

\subsection{Эстетика св.~Августина}\label{subsec:sw_august}

\y{Тексты Августина}\addcontentsline{toc}{subsubsection}{Тексты Августина}

\subsection{Условия дальнейшего развития}

\subsection{Эстетика от Боэция до Исидора}

\y{Тексты Боэция, Кассиодора и Исидора}\addcontentsline{toc}{subsubsection}{Тексты Боэция, Кассиодора и Исидора}

\subsection{Эстетика эпохи Каролингов}

\y{Тексты учёных эпохи Каролингов}\addcontentsline{toc}{subsubsection}{Тексты учёных эпохи Каролингов}

\subsection{Баланс раннехристианской эстетики}

\newpage
\part{Эстетика классического Средневековья}

\newpage
\printbibliography

\end{document}
