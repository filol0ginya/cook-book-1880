%первая глава
\subsection{Условия развития}
\z{Христианство.} Ещё до I~в. н.\,э., когда культура эллинизма была в расцвете, а Рим находился на пике силы, начало меняться отношение людей к миру и жизни, которое оторвало их от житейских дел и повернуло к трансцендентному. Во многих сферах рациональное отношение уступило место мистическому, сиюминутные потребности отошли на второй план по сравнению с религиозными. Это новое отношение и новые потребности породили новые религии, секты, обряды, религиозно-философские системы, в целом новый взгляд на мир. А с ним и новую эстетику. Из античных доктрин все материалистические и позитивистские утратили свою привлекательность, а приобрёл её платонизм. Собственным же созданием эпохи была трансцендентная, монистическая, эманационная неоплатоническая система Плотина с его экстатической теорией познания, этикой и эстетикой. 

Однако наиболее плодовитым на последствия было появление христианской религии. Почти через три века после её появления, во время которых её последователи были ещё немногочисленными и не имели влияния на власть и общественность, сохранялись старые формы жизни и мышления: эти столетия ещё относятся к античному периоду. Зато с IV~в., точнее с 313~года, когда согласно эдикта Константина Великого, христианство уже можно было беспрепятственно исповедовать, а особенно с 325~г., когда стало государственной религией, новые формы жизни и мысли взяли верх над старыми: начался новый период в истории <<обитаемого мира>>. 

Христианство опиралось на свою веру, моральный закон, принцип любви, обещание вечной жизни; ему не нужна была наука, философия, в особенности эстетика. <<Любовь к Богу "--- вот истинная философия>>, "--- говорил Иоанн Дамаскин, а Исидор Севильский писал: <<Первая задача науки "--- … Бога, а вторая "--- добродетельность жизни>>. Если же христианство должно было иметь философию, то собственную, не такую, как раньше.

Латинские Отцы Церкви в аскетичной Африке и трезвой римской обстановке желали вовсе отречься от философии. Напротив, Отцы греческой Церкви, действовавшие в Афинах или в Антиохии, в рамках философских традиций, видели, что среди язычников философия также сближается с религией: поняли потребность в философии и возможность использования создания христианской философии в согласии с древними воззрениями. Были предприняты различные попытки: Тертуллиан пробовал переложить христианскую философию наподобие стоической, Григорий Нисский "--- наподобие философии платонической, Ориген "--- неоплатонической; но Церковь не одобрила эти попытки. Однако в IV и V~вв. в письмах Греческих Отцов и Августина уже появилась установленная христианская философия, состоящая из собственной веры и тех элементов античного знания, которые были признаны Церковью. Эстетика не была в этой философии на первом месте, однако присутствовала.

Сложилось так, что первые создатели христианской философии, как Греческие Отцы, так и Августин имели существенные интересы и эстетическую компетенцию. На их эстетике, как и на всей философии, кроме античных доктрин, сложилось \emph{Св.~Писание}. Таким образом, историю эстетики того периода следует начинать с эстетических мотивов, содержащихся в \emph{Св.~Писании}.

\z{Две империи.} 
В том же IV столетии, в котором началась эра царствования христианства и которое установило правила христианской философии и эстетики, произошло другое важное изменение: разделение <<обитаемого мира>> на Восток и Запад.

Различия между восточной и западной частями Римской империи в устройстве и менталитете всегда были велики, но чрезвычайно увеличились в 395~г., когда империя политически была поделена на Восточную и Западную. С тех пор история не только политическая, но и культурная обеих частей империи пошла по своему пути. Западная империя быстро потерпела поражение, Восточная просуществовала тысячу лет. Западная подвергалась преобразованиям, консервативная Восточная сумела остановить развитие. Западная должна была хотя бы частично адаптироваться к обычаям северных завоевателей, Восточная же, находясь на рубеже Восточной Европы, оказалась подвержена влиянию Азии. А прежде всего: на Востоке могли сохраняться формы античной культуры, а на Западе они были уничтожены и забыты. Восток мог жить этими формами античной культуры, сохранять их или развивать, а Запад должен был сам создавать себе культурные формы, утратив античные. Должен был начинать с начала. Он потерял более совершенные формы, зато творил собственные. На Востоке заканчивалась античная история, на Западе начиналась новая. Если называть средневековыми те формы новой культуры, которые создавались на Западе после падения Рима, то на Востоке Средневековья не было; это было исключительно западное явление. В действительности и Восток в то время перешёл к новым формам жизни и культуры, но не начинал с начала, а продолжал античность.

Таким образом историю христианской эстетики, культуры, искусства необходимо развивать в двух направлениях: отдельно на Востоке и на Западе. А начинать надо с Востока, потому что он непосредственно связан с античностью. Здесь на протяжении многих веков греческая мысль и искусство оставались живы, Византия думала и говорила по-гречески, хотя думала также и по-христиански. До VI~в. существовала Платоновская Академия. На Востоке сохранялись многие античные традиции. Византия по замыслу Константина Великого должна была воспринять наследие Рима и действительно стала <<Новым Римом>>. В то же время благодаря историческому и географическому положению Византия была наследницей Греции. Здесь не было недостатка в античных образцах: по приказу императоров со всей империи были собраны и привезены сюда произведения античного искусства. Прямо перед церковью св.~Софии было установлено 427 греческих и римских статуй. В таких условиях очень быстро развивалось христианское искусство; тут начиналась как музыкальное, так и пластическое искусство, тут появлялись первые великие христианские храмы во главе с храмом св.~Софии. И здесь же среди христиан началась эстетическая рефлексия.