%вторая глава
\subsection{Эстетика Святого Писания}\label{sec:pismo_sw}

Те же люди, положившие начало всей христианской философии, стояли у истоков христианской эстетики: с одной стороны это были Греческие Отцы, особенно св.~Василий, а с другой стороны "--- Латинские Отцы во главе со св.~Августином. Первые имеют греческие истоки, вторые "--- римские: и те, и другие были знакомы с античным пониманием красоты и искусства и обращались к нему. Это был первый источник их эстетических воззрений "--- но вторым была их собственная христианская идеология, содержащаяся в \emph{Св.~Писании}. Хотя оно служило не эстетическим целям, в нём были обнаружены идеи и на эту тему. Особенно в \emph{Ветхом Завете}.

Слово <<красивый>> (\textgreek{kal'os}) неоднократно появляется в Септуагинте, греческом переводе \emph{Св.~Писания}. В ней были ставились и обсуждались некоторые эстетические вопросы. Из книг Ветхого Завета больше всего таких вопросов поднимают две книги, каждая из которых имеет совершенно особый характер: \emph{Книга Бытия} и \emph{Книга Премудрости}. На переднем плане красота есть также в \emph{Песни Песней}. В \emph{Екклесиасте} и \emph{Книге притч Соломоновых} о ней говорится уже реже.

\z{Книга Бытия.} \emph{Книга Бытия} уже в первом разделе содержит сообщение, которое имеет большое значение для эстетики, потому что касается красоты мира. Там говорится о том, что Бог, осматривая созданный им самим мир, оценивает свою работу. В \emph{Книге Бытия} говорится: <<И увидел Бог всё, что Он создал, и вот, хорошо весьма>>\textsuperscript{\ref{cyt:byt1:31}}. Этот оборот повторяется в \emph{Книге} не единожды, а многократно (Книга Бытия). В нём можно проследить две идеи: во-первых, уверенность в том, что мир прекрасен (уверенность в \emph{панкалии}, как говорится по-гречески), во-вторых, уверенность, что прекрасен потому, что является сознательным творением мыслящей сущности, подобно произведению искусства.

Эти мысли о прекрасном \emph{Книга Бытия} несомненно содержит в своём греческом переводе. Но похоже, что их не было в оригинале, и что они были привнесены при переводе, потому что смысл еврейского оригинала, по мнению исследователей, был другим. Греческим словом \textgreek{kal'os}, или <<прекрасный>> переводчики \emph{Септуагинты}, александрийские учёные-евреи, в III~в. до~н.\,э. перевели прилагательное в более широком значении, означающем внутренние достоинства (особенно моральные: храбрый, полезный, хороший), а также достоинства внешние, но не только эстетические. Собственный смысл слов \emph{Книги}, в которых Бог оценивает свою работу, был таков: работа завершилась у~с~п~е~ш~н~о. Эти слова содержали общую положительную оценку мира, а не только эстетическую; специальной эстетической оценки в них не было Это согласуется с общим подходом \emph{Ветхого Завета} и с тем, что красота почти не играла никакой роли в культе и библейской религиозности.

У переводчиков, помимо этого, было основание для употребления слова \textgreek{kal'on}, которое тоже имело широкий смысл, множество оттенков, означало не только эстетическую красоту, но также моральную и в целом всё, что заслуживает признания и пробуждает удовольствие. Возможно, что его использовали не имея в виду эстетической красоты, и что только впоследствии слово обрело такой смысл. Но может быть и так, что уже сами переводчики так его интерпретировали: потому что в III~в. до~н.\,э. в Александрии интеллектуальная культура была греческой и евреи тоже были подвержены греческому влиянию, которое склоняло к другому, нежели чисто моралистическому, отношению к миру.

Так или иначе, умышленно или нет, переводя библейскую идею о том, что мир удался, словом \textgreek{kal'os}, переводчики Септуагинты привнесли в Библию греческую идею о красоте мира. Даже если это не было целью их перевода, это стало его результатом. Однажды привнесённая, эта идея действовала дальше. Не прошла в латинский перевод Писания , в Вульгату, которая \textgreek{kal'on} перевела как {\sl bonum}, а не {\sl pulchrum}. Однако осталась в христианстве, в культуре Средневековья и Нового времени.

Эта христианская эстетика, провозглашающая красоту мира, хотя основывалась на Ветхом Завете, имела иной источник; нельзя даже утверждать, что она имела два источника, греческий и библейский, потому что целиком была греческой. То, что кажется библейской эстетикой, было греческого происхождения, попало в Библию под греческим влиянием, посредством перевода на греческий.

Идея о красоте творения в том виде, в каком  она появляется в Книге Бытия, возвращается в Книге Премудрости (13, 7 и 13, 5), Екклесиасте (43, 9 и 39, 16)\textsuperscript{\ref{cyt:tyk39:21}}, где деяния Иеговы в природе и истории называются по-гречески \textgreek{kal'a}. Та же идея Септуагинты существует в Екклесиасте (3, 11)\textsuperscript{\ref{cyt:ta3:11}}, а также в Псалме 25, 8: <<Господи, я влюбился в красоту дома Твоего>>\textsuperscript{\ref{cyt:ps25:8}}. Также несколько иными словами в Псалме 95, 5\textsuperscript{\ref{cyt:ps95:5}}; где используется слово \textgreek{<wra\~ios}, имеющее более эстетической значение, чем \textgreek{kal'os}. Во всех этих местах \emph{Св.~Писания} можно разглядеть отзвук эстетических воззрений эллинизма.

Все эти книги Ветхого Завета появились в эллинистический период: Екклесиаст появился в III~в. до~н.\,э., \emph{Книга притч Соломоновых} в начале II~в., а \emph{Книга Премудрости}, даже только в I~в. до~н.\,э., а значит, в период, когда евреи-теологи, такие как Филон Александрийский, хорошо знали эллинистическую философию.

\z{Книга Премудрости.} \emph{Книга Премудрости} "--- это та книга \emph{Ветхого Завета}, в которой чаще всего упоминается красота. Она проповедовала красоту творения, видела в этой красоте доказательство существования и деятельности Бога; через величие и красоту творения познаётся творец, вызвавший их к жизни (13, 5)\textsuperscript{\ref{cyt:prem13:5}}. Говорила о  красоте не только божественных творений, но и человеческих, не только природы, но и искусства, очарование которых так велико, что <<люди приписывают им божественность>>.

Но кроме того эта книга привнесла совершенно иной, не религиозный, а философский мотив, чисто греческий, пифагорейско-платоновский.  Она утверждала, что именно Бог устроил <<всё в соответствии с мерой, числом и весом>> {\sl omnia in mensura et numero et pondere} (11, 21)\textsuperscript{\ref{cyt:prem11:21}}. Провозгласила математическо-эстетическую теорию. Такая теория в религиозной книге была уже особым проявлением греческого влияния: в данном случае не только на перевод, но и на саму книгу. Тот факт, что эта теория оказалась на страницах \emph{Св.~Писания}, для средневековой эстетики имел огромное значение, авторитет \emph{Писания} позволил учёным утвердить её и вызвал это неожиданное явление, что математическая теория стала одной из главных эстетических теорий религиозного периода. Эта идея не единожды появляется в \emph{Ветхом Завете}: также в \emph{Книге притч Соломоновых} сказано, что Бог творение своё посчитал и измерил, {\sl denumeravit et mensus est} (1, 9)\textsuperscript{\ref{cyt:tyk1:9}}.

\z{Екклесиаст и Песнь песней.}
В то время как благодаря грекам в эстетику \emph{Св.~Писания} вошли оптимистические и математические мотивы, то от самих израильтян в Ветхом Завете появилось нечто совсем другое: собственное б~е~з~р~а~з~л~и~ч~н~о~е отношение к прекрасному и в целом к внешнему виду вещей. Рассказывая о зданиях, израильтяне описывали, как те были построены, но никогда "--- как выглядели; о людях "--- Иосифе, Давиде или Авессаломе "--- правдиво писали, что они прекрасны, но не описывали их красоты. Не проявляли интереса к внешнему виду предметов и людей, их мысль не останавливалась на нём, как если бы ускользал от их внимания; если говоря о людях, обращали внимание на особенности внешности, то только на те, которые выражали внутренние переживания.

От безразличного отношения было недалеко до нежелания. Книга притч Соломоновых выразила убеждённость в тщетности красоты: {\sl Fallax gratia et vana est pulchritudo} (31, 30)\textsuperscript{\ref{cyt:prit31:30}}. Можно было бы и здесь усматривать греческий мотив, появившийся из скептических кругов греческой философии; но презрительного отношения к красоте было в \emph{Библии} больше, чем в греческой философии. Примечательно, что о чувственной, зримой, эстетической красоте \emph{Св. Писание} говорит, описывая древо познания добра и зла, которое \emph{Вульгата} называет {\sl pulchrum oculis aspectuque delectabile}, <<прекрасным для очей и приятным для взора>>: а значит в случае, когда шла речь о чём-то угрожающем, об источнике человеческих бед.

Это негативное отношение \emph{Ветхого Завета} к красоте не нашёл широкого резонанса среди христиан; не помешал тому, что среди них были также те, кто преклонялся перед красотой, видел в ней благо, данное Богом, свидетельство Его совершенства. Две противоположных позиции по отношению к красоте "--- тщетность красоты и красота как признак божественности "--- будут постоянно на протяжении веков проявляться в христианской эстетике.

Другой особенностью отношения израильтян к внешнему виду вещей было то, что его трактовали как с~и~м~в~о~л. Они были уверены, что видимое не является важным само по себе, но только как знак невидимого. О красоте человека являются его свойства в~н~у~т~р~е~н~н~и~е, проявляющиеся в его облике. Если израильтяне вырезали или рисовали своих пророков, жертву Исаака или Моисея в горящем кусте (они делали это редко, но делали, как показывают раскопки III\,в. в Dura Europos на Евфрате), то для того, чтобы представить деятельность Бога; язычники рисовали и вырезали своих богов, они же "--- символы и деяния своего Бога. Христиане частично переняли их воззрения, но переняли также и античные воззрения. И благодаря этому их отношение к красоте было двояким: непосредственный и символичный, и оба проявлялись в их эстетике.

Ещё одна особенность понимания израильтянами красоты нашла выражение в \emph{Песни Песней}, в содержащемся в ней описании красоты невесты. Там указаны двоякие черты, описывающие красоту: с одной стороны такие, как моральная чистота и недоступность, которые сравниваются с башней и крепостью: они являются внутренними достоинствами, проявляющимися внешне. Но с другой стороны, невеста имеет достоинства, составляющие её очарование, которые сравниваются с цветами, украшениями, с тем, что приятно на вкус и запах, со сладостью вина, с благовониями Ливана, шафраном, алоэ, с источником пресной воды. 

Как первые, так и вторые черты изображают иное, чем у греков, понимание прекрасного.

\z{Еврейское и греческое понимание прекрасного.}
1.~Вещи имели для греков н~е~п~о~с~р~е~д~с~т~в~е~н~н~у~ю красоту, а в \emph{Ветхом Завете} "--- опосредованную, символическую. 2.~Для греков о прекрасном являлись о~с~о~б~е~н~н~о~с~т~и вещей, тут "--- их действия, в~п~е~ч~а~т~л~е~н~и~я, которые они производят. 3.~Для греков прекрасное было, как правило, в~и~з~у~а~л~ь~н~о~е, тут "--- в той же, если не в большей степени, было предметом других смыслов, знаков, запахов, звуков; было одновременно z ponętą zmysłową, которая в других смыслах является не менее, если не более сильной. А также 4.~для греков красота была гармонией, то есть гармоничным р~а~с~п~о~л~о~ж~е~н~и~е~м составляющих частей, тут "--- было свойством отдельных элементов; для греков лежало в сочетании предметов, тут "--- в их разобщённости, красивым было прежде всего то, что не смешано, чисто. Одноцветные и светящиеся солнце и месяц были для израильтян \emph{Ветхого Завета} красивей, чем какое-либо сочетание цветов, и аналогично дело обстояло с музыкой. 5.~Таким образом, когда у греков красивой являлась ф~о~р~м~а, то тут "--- и~н~т~е~н~с~и~в~н~о~с~т~ь свойства, цвета, света, запаха, звуков. Красота содержалась для израильтян в том, что живёт и творит, we wdzięku i sile, а не совершенной пропорции, не в форме. <<Величаюшую красоту израильтяне находили в бесформенном и пугающем огне и животворящем свете>>. 6.~В то время, как греки были чувствительны к ~ц~в~е~т~у вне формы, израильтяне скорей к ~с~в~е~т~у; были более чувствительны к яркости света, чем к насыщенности цвета. И отношение к цвету имели другое: в то время, как можно было предполагать, что для греков прекраснейшим цветом был голубой, цвет неба и глаз Афины, то израильтяне не имели для него даже, как утверждают филологи, непосредственного названия; для них красивым был красный цвет. Далее: 7.~Красота классического греческого периода было с~т~а~т~и~ч~н~ы~м, было красотой покоя и равновесия, для израильтян же красота с самого начала была динамичной, была красота движения, жизни, действия. 8.~У греков основной идеей было, что красота есть в природе; у израильтян красота природы играла незначительную  роль. 9.~Греки вырезали своих богов, а у израильтян существовал запрет на изображение Бога. Поскольку понимали Его не образно, красота в прямом значении слова не могла быть Его свойством. Действительно, сказано в \emph{Писании}, что Бог создал человека <<по образу и подобию своему>>, но это {\sl imago Dei} понималось не как воссоздание телесного внешнего вида Бога, но как телесный образ бестелесного Бога, это {\sl imago} было тут формой откровения, а не подобия.

Бога \emph{Ветхого Завета} отличали самые высокие атрибуты, среди которых было величие или великолепие, но не было --- красоты. И всё же хотя и не в \emph{Книгах Моисея} \emph{Ветхого Завета}, а в \emph{Песни Песней} есть такое предложение: {\sl ostende mihi faciem tuam… facies tua speciosa}\textsuperscript{\ref{cyt:prit2:14}}. С господствующим среди израильтян взглядом это предложение удаётся согласовать, только если принять, что слово {\sl speciosus} "--- <<красивый>> использовано в другом значении, а именно в таком, которое могло бы относиться к божеству: красивый, как пробуждающий не чувственную, но исключительно умственную привлекательность. В переносном значении говорил также о красоте Бога мыслитель, близко с \emph{Ветхим Заветом} связанный, Филон Александрийский; видение того, что не сотворённое и божественное "--- писал он "--- есть лучшее из хорошего и прекраснейшее из прекрасного\textsuperscript{\ref{cyt:gaj5}}. Это сублимированное понятие прекрасного закрепилось в христианской эстетике.

Понимание прекрасного, которому даёт определение \emph{Ветхий Завет}, имело, скорее всего, не единственный источник: появилось из условий жизни израильтян, из их монотеистической религии, а в особенности из запретов, которые эта религия привносила.

\z{Запрет на изображения.}
Моисей запретил изображать Бога; даже больше, даже каких-либо живых существ. В \emph{Книгах Моисеевых} этот запрет был сформулирован максимально решительным образом и повторяется не менее 8 раз (Исх. 20, Исх. 20, 23; Исх. 34, 17; Лев. 26, 1; Втор. 4, 15; Втор. 4, 23; Втор. 5, 8; Втор. 27, 15). Шесть раз сказано, что нельзя robić bogów, один раз --- что нельзя создавать никаких скульптурных подобий и ничего <<na kształt mężczyzny lub niewiasty>>, четырежды "--- не ваять каких-либо живых существ, никаких подобий тому, что есть наверху в небе и что внизу на земле, ни тому, что в воде под землёй\textsuperscript{\ref{cyt:ish20:4}}. Запрещались <<подобия>> вообще, но особенно скульптуры, литые, вырезанные или кованые; один раз упоминаются оба вида, запрещая скульптуры как литые, так и вырезанные (Втор. 27, 15)\textsuperscript{\ref{cyt:wtor27:15}}.

Смысл этих запретов несомненный: имели религиозный характер, были введены для того, чтобы предотвратить идолопоклонство. Их радикализм был особенный: он включал в себя все живые существа. А также результат этих запретов: их скрупулёзно соблюдали на протяжении столетий. Они привели к тому, что у израильтян не было ни скульпторов, ни художников, что они по сути перестали заниматься изобразительным искусством. Дальнейшим их результатом было то, что эстетические потребности народа уменьшились. А если и выражались, то не в изящные формы, но в богатстве материи. \emph{Иезекиль} пишет (28, 13): <<твои одежды были украшены всякими драгоценными камнями; рубин, топаз и алмаз, хризолит, оникс, яспис, сапфир, карбункул и изумруд и золото>>. Драгоценность и великолепие были для израильтян самой большой красотой.

\z{Античное наследие.}
Подводя итог, следует сказать следующее: т~р~и мотива, связанные с эстетикой, имели в \emph{Ветхом Завете} большое значение; во-первых, красота вселенной, во-вторых, происхождение красоты из <<меры, числа и веса>>, в-третьих, тщетность и даже небезопасность красоты. Итак, все три были известны грекам: первый был эллинистическим мотивом п~а~н~к~а~л~и~и, второй "--- пифагорейским мотивом м~е~р~ы, третий "--- мотивом к~и~н~и~ч~е~с~к~и~м. Два первых, по всей видимости, попали в \emph{Св.~Писание} от греков, и только до третьего автор \emph{Екклесиаста} додумался сам.

Мотив меры некоторые историки называют <<мотивом мудрости>>, или мотивом \emph{Книги Премудрости}, но она взяла его несомненно из античности, не придумала сама. Скорей мотив \emph{панкалии}, хоть тоже начавшийся в античности, может быть тесней связан с \emph{Библией}, потому что получил в ней значение, которого не имел у греков, и мог бы с большей вероятностью  выступать в качестве <<библейского>> мотива.

Эстетика христиан черпала из обоих источников: из \emph{Ветхого Завета} и греческих авторов. То, что \emph{Ветхий Завет} "--- в некоторых своих книгах, а особенно в переводе \emph{Септуагинты} "--- сам черпал из греков, облегчало объединение обоих источников. Однако двойственность, напряжённость и противоречия остались. Эстетика христиан знала красоту символическую, как непосредственную, красоту света, как красоту гармонии, красоту жизни, как красоту покоя, видела в красоте {\sl vana pulchritudo}, это снова одно из наивысших совершенств мира. Тертуллиан выступал в поддержку сохранения запрета изготовления подобий (однако с другим доводом: чтобы избежать лжи, которая есть в каждом воспроизведении), но большая часть христиан пошла вслед за греками, занималась искусством и представляла в них не только творение, но и самого Творца. Двойственность источников христианства отразилась главным образом в практике, во вкусе, удовольствиях, произведениях искусства, и меньше в теории, в научных обобщениях, потому что в них христиане следовали за античностью.

Ранние христиане жили в свете эллинистическом и если они ставили перед собой эстетические вопросы, то руководствовались при этом эллинистическими понятиями; многие из тех понятий в их эпохе были уже действующими понятиями. Более образованные знали также теории греческих учёных, как распространённый эклектиками взгляд, что красота "--- в расположении частей, а также как новая теория Плотина, что красота "--- в свете и блеске.

Но перенятые из античности эстетические взгляды получили у христиан другое основание, наполнились другим значением. Это произошло благодаря их религиозному отношению, всех значений, относящихся к Богу, и моральному отношению, подчиняющему все задачи человека морали. Мир прекрасен "--- потому что его создал Бог. Мир измерен и учтён "--- потому что это сделал Бог. Красота тщетна "--- по отношению к вечности и моральных задач, стоящих перед людьми. С учётом этих предположений переход от эстетики античной к христианской "--- хотя христиане переняли основные мысли греков и римлян "--- создало новые мотивы: не подробными идеями, но посредством введения нового взгляда на мир.

\z{Евангелие.}
Мировоззрение христиан опиралось прежде всего на \emph{Новый Завет}. Но он содержал ещё меньше эстетических мотивов, чем \emph{Ветхий}; можно даже утверждать, что их там вовсе не было. В действительности слово <<прекрасный>> (\textgreek{kal'os}) встречается там неоднократно: \emph{Евангелие от Матфея} говорит, что дерево дало прекрасные плоды (7, 17; 12, 33), что сеятель сеет прекрасное зерно (13, 27; 37, 38), а особенно "--- прекрасными являются действия (5,16). Впрочем, красота, о которой тут говорится, не является эстетической красотой, но всегда и исключительно моральной; является красотой и добром в христианском понимании, то есть в значении совершенного действия любви и веры. Когда евангелист св.~Иоанн говорит о \textgreek{<o poim`hn <o kal'os} (10, 11 и 14), то можно это переводить только как <<добрый пастырь>>, а не <<красивый>>. Так же и в ранней христианской литературе \textgreek{kal`os n'omos} употребляется в смысле <<доброго закона>>, а  \textgreek{kal`os di'akonos} в смысле <<доброго священника>>.

Из всех разновидностей прекрасного, которые знал эллинизм, \emph{Евангелие} высоко ставило только одну: красоту в моральном значении. Красота в чисто эстетическом смысле, красота внешнего вида или формы, не была для него существенной. Но однако её не обошло стороной, не пренебрегло красивым <<украшением>> предметов; в \emph{Нагорной проповеди}  (Мф., 6, 28—29) сказано: <<Посмотрите на полевые лилии, как они растут: ни трудятся, ни прядут; но говорю вам, что и Соломон во всей славе своей не одевался так, как всякая из них>>\textsuperscript{\ref{cyt:mat6:28}}. Мир телесный и его красота имеет свой вес, потому что как говорит св.~Павел, через него <<wiekuista moc i bóstwo… dla umysłu widzialnymi się stały>>\textsuperscript{\ref{cyt:rim1:20}}.

Ранние христиане не находили в \emph{Евангелии} особенных эстетических утверждений; однако находили в нём подсказку, какую позицию занимать в отношении каждой сферы жизни, а значит также в отношении красоты и искусства. Эта позиция опиралась на убеждение о превосходстве вечных благ над временными, духовных над телесными, моральных над всеми остальными. Не было в \emph{Новом Завете} эстетических теорий, но был пример, какие эстетические теории христиане могут считать своими. И немного времени прошло, и христианские мыслители воспользовались этим примером.